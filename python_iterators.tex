%%%%%%%%%%%%%%%%%%%%%%%%%%%%%%%%%%%%%%%%%%%%%%%%%%%%%%%%%%%%%%%%%%%%%%%%%%%%%%%%%%
\begin{frame}[fragile]\frametitle{}
\begin{center}
{\Large Iterators}
\end{center}
\end{frame}

%%%%%%%%%%%%%%%%%%%%%%%%%%%%%%%%%%%%%%%%%%%%%%%%%%%%%%%%%%%%%%%%%%%%%%%%%%%%%%%%%%%
\begin{frame}[fragile]\frametitle{Usage of ``for'' loop}
Looping over a list:
\begin{lstlisting}
>>> for i in [1, 2, 3, 4]:
...     print i
\end{lstlisting}

With a string, it loops over its characters:
\begin{lstlisting}
>>> for c in "python":
...     print c
\end{lstlisting}
\end{frame}

%%%%%%%%%%%%%%%%%%%%%%%%%%%%%%%%%%%%%%%%%%%%%%%%%%%%%%%%%%%%%%%%%%%%%%%%%%%%%%%%%%%
\begin{frame}[fragile]\frametitle{Usage of ``for'' loop}
With a dictionary, it loops over its keys:
\begin{lstlisting}
>>> for k in {"x": 1, "y": 2}:
...     print k
\end{lstlisting}

With a file, it loops over lines of the file:
\begin{lstlisting}
>>> for line in open("a.txt"):
...     print line
\end{lstlisting}
\end{frame}

%%%%%%%%%%%%%%%%%%%%%%%%%%%%%%%%%%%%%%%%%%%%%%%%%%%%%%%%%%%%%%%%%%%%%%%%%%%%%%%%%%%
\begin{frame}[fragile]\frametitle{Iterables}
    \begin{itemize}
    \item  So there are many types of objects which can be used with a for loop. 
    \item These are called iterable objects.
	\item There are many functions which consume these iterables:
	\begin{lstlisting}
>>> ",".join(["a", "b", "c"])
'a,b,c'
>>> ",".join({"x": 1, "y": 2})
'y,x'
>>> list("python")
['p', 'y', 't', 'h', 'o', 'n']
>>> list({"x": 1, "y": 2})
['y', 'x']
\end{lstlisting}
    \end{itemize}
\end{frame}

%%%%%%%%%%%%%%%%%%%%%%%%%%%%%%%%%%%%%%%%%%%%%%%%%%%%%%%%%%%%%%%%%%%%%%%%%%%%%%%%%%%
\begin{frame}[fragile]\frametitle{Iteration}
    \begin{itemize}
    \item  The act of going over a collection
    \item  Explicit in the form of `for'
    \item  Implicit in many forms: list(), tuple(), dict(), map(), reduce(), zip()
    \item Used with `in'
    \end{itemize}
\end{frame}

%%%%%%%%%%%%%%%%%%%%%%%%%%%%%%%%%%%%%%%%%%%%%%%%%%%%%%%%%%%%%%%%%%%%%%%%%%%%%%%%%%%
\begin{frame}[fragile]\frametitle{Iteration}
    \begin{itemize}
    \item  An iterator is an object adhering to the iterator protocol : basically this means that it has a next method, which, when called, returns the next item in the sequence, 
    \item and when there's nothing to return, raises the StopIteration exception.
\item 
An iterator object allows to loop just once.
\item  It holds the state (position) of a single iteration, or from the other side, each loop over a sequence requires a single iterator object. 
\item This means that we can iterate over the same sequence more than once concurrently. 
\item Separating the iteration logic from the sequence allows us to have more than one way of iteration.
    \end{itemize}
\end{frame}

%%%%%%%%%%%%%%%%%%%%%%%%%%%%%%%%%%%%%%%%%%%%%%%%%%%%%%%%%%%%%%%%%%%%%%%%%%%%%%%%%%%
\begin{frame}[fragile]\frametitle{Iteration}
    \begin{itemize}
\item Calling the \_\_iter\_\_ method on a container to create an iterator object is the most straightforward way to get hold of an iterator. 
\item The iter function does that for us, saving a few keystrokes.
\item When used in a loop, StopIteration is swallowed and causes the loop to finish. 
\item But with explicit invocation, we can see that once the iterator is exhausted, accessing it raises an exception.
    \end{itemize}
\end{frame}

%%%%%%%%%%%%%%%%%%%%%%%%%%%%%%%%%%%%%%%%%%%%%%%%%%%%%%%%%%%%%%%%%%%%%%%%%%%%%%%%%%%
\begin{frame}[fragile]\frametitle{Iteration}
    \begin{itemize}
\item Support for iteration is pervasive in Python: all sequences and unordered containers in the standard library allow this. 
\item The concept is also stretched to other things: e.g. file objects support iteration over lines.
\item The file is an iterator itself and it's \_\_iter\_\_ method doesn't create a separate object: only a single thread of sequential access is allowed.
    \end{itemize}
\end{frame}



%%%%%%%%%%%%%%%%%%%%%%%%%%%%%%%%%%%%%%%%%%%%%%%%%%%%%%%%%%%%%%%%%%%%%%%%%%%%%%%%%%%
\begin{frame}[fragile]\frametitle{Iteration Protocol}
    \begin{itemize}
    \item Protocol of 2 methods:
    \begin{itemize}
    	\item \_\_iter\_\_(): get iterator
	\item next(): get next value, raises StopIteration when done

        \end{itemize}
        \item Explicit iterator creation with iter() 

    \end{itemize}
\end{frame}

%%%%%%%%%%%%%%%%%%%%%%%%%%%%%%%%%%%%%%%%%%%%%%%%%%%%%%%%%%%%%%%%%%%%%%%%%%%%%%%%%%%
\begin{frame}[fragile]\frametitle{Iteration Protocol}
The built-in function iter takes an iterable object and returns an iterator.
	\begin{lstlisting}
>>> x = iter([1, 2, 3])
>>> x
<listiterator object at 0x1004ca850>
>>> x.next()
1
>>> x.next()
2
>>> x.next()
3
\end{lstlisting}
\end{frame}

%%%%%%%%%%%%%%%%%%%%%%%%%%%%%%%%%%%%%%%%%%%%%%%%%%%%%%%%%%%%%%%%%%%%%%%%%%%%%%%%%%%
\begin{frame}[fragile]\frametitle{Iteration Protocol}

\begin{lstlisting}
>>> x.next()
2
>>> x.next()
3
>>> x.next()
Traceback (most recent call last):
  File "<stdin>", line 1, in <module>
StopIteration
\end{lstlisting}
Each time we call the next method on the iterator gives us the next element. If there are no more elements, it raises a StopIteration.
\end{frame}


%%%%%%%%%%%%%%%%%%%%%%%%%%%%%%%%%%%%%%%%%%%%%%%%%%%%%%%%%%%%%%%%%%%%%%%%%%%%%%%%%%%
\begin{frame}[fragile]\frametitle{Reading Lines}
Without Iterator:
\begin{lstlisting}
with open('data.txt', 'r') as f:
    while True:
        line = f.readline()
        if not line: 
            break
        else: 
            print(line.rstrip())
\end{lstlisting}

\end{frame}

%%%%%%%%%%%%%%%%%%%%%%%%%%%%%%%%%%%%%%%%%%%%%%%%%%%%%%%%%%%%%%%%%%%%%%%%%%%%%%%%%%%
\begin{frame}[fragile]\frametitle{Reading Lines}
With Iterator:
\begin{lstlisting}
with open('data.txt', 'r') as f:
    for line in f:
        print(line.rstrip())
\end{lstlisting}

\end{frame}


%%%%%%%%%%%%%%%%%%%%%%%%%%%%%%%%%%%%%%%%%%%%%%%%%%%%%%%%%%%%%%%%%%%%%%%%%%%%%%%%%%%
\begin{frame}[fragile]\frametitle{Iterator Class}
    \begin{itemize}
    \item Iterators are implemented as classes. 
    \item Here is an iterator that works like built-in range function.
	\begin{lstlisting}
class yrange:
    def __init__(self, n):
        self.i = 0
        self.n = n

    def __iter__(self):
        return self

    def next(self):
        if self.i < self.n:
            i = self.i
            self.i += 1
            return i
        else:
            raise StopIteration()
\end{lstlisting}
    \end{itemize}
\end{frame}

%%%%%%%%%%%%%%%%%%%%%%%%%%%%%%%%%%%%%%%%%%%%%%%%%%%%%%%%%%%%%%%%%%%%%%%%%%%%%%%%%%%
\begin{frame}[fragile]\frametitle{Iterator Class}
    \begin{itemize}
    \item The \_\_iter\_\_ method is what makes an object iterable. Behind the scenes, the iter function calls \_\_iter\_\_ method on the given object.
    \item The return value of \_\_iter\_\_ is an iterator.
    \item It should have a next method and raise StopIteration when there are no more elements.
	\begin{lstlisting}
>>> y = yrange(3)
>>> y.next()
0
>>> y.next()
1
>>> y.next()
2
>>> y.next()
Traceback (most recent call last):
  File "<stdin>", line 1, in <module>
  File "<stdin>", line 14, in next
StopIteration
\end{lstlisting}
    \end{itemize}
\end{frame}

%%%%%%%%%%%%%%%%%%%%%%%%%%%%%%%%%%%%%%%%%%%%%%%%%%%%%%%%%%%%%%%%%%%%%%%%%%%%%%%%%%%
\begin{frame}[fragile]\frametitle{Iteration Usage}
Many built-in functions accept iterators as arguments.
\begin{lstlisting}
>>> list(yrange(5))
[0, 1, 2, 3, 4]
>>> sum(yrange(5))
10
\end{lstlisting}
In the above case, both the iterable and iterator are the same object. Notice that the \_\_iter\_\_ method returned self. It need not be the case always.
\end{frame}

%%%%%%%%%%%%%%%%%%%%%%%%%%%%%%%%%%%%%%%%%%%%%%%%%%%%%%%%%%%%%%%%%%%%%%%%%%%%%%%%%%%
\begin{frame}[fragile]\frametitle{Iteration Usage}
Iterable and Iterator can be different
\begin{lstlisting}
class zrange:
    def __init__(self, n):
        self.n = n

    def __iter__(self):
        return zrange_iter(self.n)

\end{lstlisting}
\end{frame}

%%%%%%%%%%%%%%%%%%%%%%%%%%%%%%%%%%%%%%%%%%%%%%%%%%%%%%%%%%%%%%%%%%%%%%%%%%%%%%%%%%%
\begin{frame}[fragile]\frametitle{Iteration Usage}
Iterable and Iterator can be different
\begin{lstlisting}
class zrange_iter:
    def __init__(self, n):
        self.i = 0
        self.n = n

    def __iter__(self):
        # Iterators are iterables too.
        # Adding this functions to make them so.
        return self

    def next(self):
        if self.i < self.n:
            i = self.i
            self.i += 1
            return i
        else:
            raise StopIteration()
\end{lstlisting}
\end{frame}


%%%%%%%%%%%%%%%%%%%%%%%%%%%%%%%%%%%%%%%%%%%%%%%%%%%%%%%%%%%%%%%%%%%%%%%%%%%%%%%%%%%
\begin{frame}[fragile]\frametitle{Iteration Usage}
If both iteratable and iterator are the same object, it is consumed in a single iteration.
\begin{lstlisting}
>>> y = yrange(5)
>>> list(y)
[0, 1, 2, 3, 4]
>>> list(y)
[]
>>> z = zrange(5)
>>> list(z)
[0, 1, 2, 3, 4]
>>> list(z)
[0, 1, 2, 3, 4]
\end{lstlisting}
\end{frame}



%%%%%%%%%%%%%%%%%%%%%%%%%%%%%%%%%%%%%%%%%%%%%%%%%%%%%%%%%%%%%%%%%%%%%%%%%%%%%%%%%%%
\begin{frame}[fragile]\frametitle{Iteration Advantages}
    \begin{itemize}
    \item Cleaner code
    \item     Iterators can work with infinite sequences
    \item     Iterators save resources
    \end{itemize}
   \end{frame}
   
%%%%%%%%%%%%%%%%%%%%%%%%%%%%%%%%%%%%%%%%%%%%%%%%%%%%%%%%%%%%%%%%%%%%%%%%%%%%%%%%%%%
\begin{frame}[fragile]\frametitle{Iteration Assinment}
Write an iterator class reverse\_iter, that takes a list and iterates it from the reverse direction:
\begin{lstlisting}
>>> it = reverse_iter([1, 2, 3, 4])
>>> it.next()
4
>>> it.next()
3
>>> it.next()
2
>>> it.next()
1
>>> it.next()
Traceback (most recent call last):
  File "<stdin>", line 1, in <module>
StopIteration
\end{lstlisting}
\end{frame}





