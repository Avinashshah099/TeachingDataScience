%%%%%%%%%%%%%%%%%%%%%%%%%%%%%%%%%%%%%%%%%%%%%%%%%%%%%%%%%%%%%%%%%%%%%%%%%%%%%%%%%%
\begin{frame}[fragile]\frametitle{}
\begin{center}
{\Large Decorators}
\end{center}
\end{frame}

%%%%%%%%%%%%%%%%%%%%%%%%%%%%%%%%%%%%%%%%%%%%%%%%%%%%%%%%%%%%%%%%%%%%%%%%%%%%%%%%%%%
\begin{frame}[fragile]\frametitle{Decorators Intro}
``This amazing feature appeared in the language almost apologetically and with concern that it might not be that useful.''
-  Bruce Eckel in ``An Introduction to Python Decorators''
\end{frame}

%%%%%%%%%%%%%%%%%%%%%%%%%%%%%%%%%%%%%%%%%%%%%%%%%%%%%%%%%%%%%%%%%%%%%%%%%%%%%%%%%%%
\begin{frame}[fragile]\frametitle{Decorators Intro}
        \begin{lstlisting}
def my_decorator(some_function):
    def wrapper():
        print("Something is happening before some_function() is called.")
        some_function()
        print("Something is happening after some_function() is called.")
    return wrapper

def just_some_function():
    print("Wheee!")

just_some_function = my_decorator(just_some_function)
just_some_function()

Something is happening before some_function() is called.
Wheee!
Something is happening after some_function() is called.
\end{lstlisting}
\end{frame}



%%%%%%%%%%%%%%%%%%%%%%%%%%%%%%%%%%%%%%%%%%%%%%%%%%%%%%%%%%%%%%%%%%%%%%%%%%%%%%%%%%%
\begin{frame}[fragile]\frametitle{Decorators Intro}
    \begin{itemize}
    	\item Decorator is way to dynamically add some new behavior to some objects. 
    \item  Since functions and classes are objects, they can be passed around. 
    \item Since they are mutable objects, they can be modified. 
    \item The act of altering a function or class object after it has been constructed but before it is bound to its name is called decorating.
    \end{itemize}
\end{frame}

%%%%%%%%%%%%%%%%%%%%%%%%%%%%%%%%%%%%%%%%%%%%%%%%%%%%%%%%%%%%%%%%%%%%%%%%%%%%%%%%%%%
\begin{frame}[fragile]\frametitle{Decorators Intro}
        \begin{lstlisting}
def my_decorator(some_function):
    def wrapper():
        num = 10
        if num == 10:
            print("Yes!")
        else:
            print("No!")
        some_function()
        print("Something is happening after some_function() is called.")
    return wrapper

def just_some_function():
    print("Wheee!")

just_some_function = my_decorator(just_some_function)
just_some_function()

Yes!
Wheee!
Something is happening after some_function() is called.
\end{lstlisting}
\end{frame}

%%%%%%%%%%%%%%%%%%%%%%%%%%%%%%%%%%%%%%%%%%%%%%%%%%%%%%%%%%%%%%%%%%%%%%%%%%%%%%%%%%%
\begin{frame}[fragile]\frametitle{Decorators Intro}
Python allows you to simplify the calling of decorators using the @ symbol (this is called “pie” syntax).
        \begin{lstlisting}
@my_decorator
def just_some_function():
    print("Wheee!")

just_some_function()

Yes!
Wheee!
Something is happening after some_function() is called.
\end{lstlisting}
\end{frame}

%%%%%%%%%%%%%%%%%%%%%%%%%%%%%%%%%%%%%%%%%%%%%%%%%%%%%%%%%%%%%%%%%%%%%%%%%%%%%%%%%%%
\begin{frame}[fragile]\frametitle{Decorators Intro}
There are two things hiding behind the name ``decorator'':
    \begin{itemize}
    \item The function which does the work of decorating, i.e. performs the real work
    \item The expression adhering to the decorator syntax, i.e. an at-symbol and the name of the decorating function.
    \end{itemize}
        \begin{lstlisting}
@decorator             # [2]
def function():        # [1]
    pass
\end{lstlisting}
    \begin{itemize}
    \item [1] A function is defined in the standard way.
    \item [2] An expression starting with @ placed before the function definition is the decorator.
    \end{itemize}
\end{frame}

%%%%%%%%%%%%%%%%%%%%%%%%%%%%%%%%%%%%%%%%%%%%%%%%%%%%%%%%%%%%%%%%%%%%%%%%%%%%%%%%%%%
\begin{frame}[fragile]\frametitle{Decorators Intro}
    \begin{itemize}
    \item The part after @ must be a simple expression, usually this is just the name of a function or class. 
    \item This part is evaluated first, and after the function defined below is ready,
    \item The decorator is called with the newly defined function object as the single argument. 
    \item The value returned by the decorator is attached to the original name of the function.
    \end{itemize}
\end{frame}

%%%%%%%%%%%%%%%%%%%%%%%%%%%%%%%%%%%%%%%%%%%%%%%%%%%%%%%%%%%%%%%%%%%%%%%%%%%%%%%%%%%
\begin{frame}[fragile]\frametitle{Decorators Intro}
    \begin{itemize}
    \item Decorators can be applied to functions and to classes. 
    \item For classes the semantics are identical : the original class definition is used as an argument to call the decorator and 
    \item whatever is returned is assigned under the original name.
    \end{itemize}
\end{frame}

%%%%%%%%%%%%%%%%%%%%%%%%%%%%%%%%%%%%%%%%%%%%%%%%%%%%%%%%%%%%%%%%%%%%%%%%%%%%%%%%%%%
\begin{frame}[fragile]\frametitle{Decorators Example}
In the example we will create a simple example which will print some statement before and after the execution of a function.
        \begin{lstlisting}
>>> def my_decorator(func):
...     def wrapper(*args, **kwargs):
...         print("Before call")
...         result = func(*args, **kwargs)
...         print("After call")
...         return result
...     return wrapper
...
>>> @my_decorator
... def add(a, b):
...     "Our add function"
...     return a + b
...
>>> add(1, 3)
Before call
After call
4
\end{lstlisting}
\end{frame}



%%%%%%%%%%%%%%%%%%%%%%%%%%%%%%%%%%%%%%%%%%%%%%%%%%%%%%%%%%%%%%%%%%%%%%%%%%%%%%%%%%%
\begin{frame}[fragile]\frametitle{Decorators Example}
        \begin{lstlisting}
import time
def timing_function(some_function):
    def wrapper():
        t1 = time.time()
        some_function()
        t2 = time.time()
        return "Time it took to run the function: " + str((t2 - t1)) + "\n"
    return wrapper

@timing_function
def my_function():
    num_list = []
    for num in (range(0, 10000)):
        num_list.append(num)
    print("\nSum of all the numbers: " + str((sum(num_list))))

print(my_function())
\end{lstlisting}
\end{frame}



%%%%%%%%%%%%%%%%%%%%%%%%%%%%%%%%%%%%%%%%%%%%%%%%%%%%%%%%%%%%%%%%%%%%%%%%%%%%%%%%%%%
\begin{frame}[fragile]\frametitle{Decorators Intro}
    \begin{itemize}
    \item Decorators can be stacked : the order of application is bottom-to-top, or inside-out. 
    \item The semantics are such that the originally defined function is used as an argument for the first decorator, whatever is returned by the first decorator is used as an argument for the second decorator,
    \item  and whatever is returned by the last decorator is attached under the name of the original function.
    \end{itemize}
\end{frame}

%%%%%%%%%%%%%%%%%%%%%%%%%%%%%%%%%%%%%%%%%%%%%%%%%%%%%%%%%%%%%%%%%%%%%%%%%%%%%%%%%%%
\begin{frame}[fragile]\frametitle{Decorators Intro}
    \begin{itemize}
    \item The decorator syntax was chosen for its readability. 
    \item Since the decorator is specified before the header of the function, it is obvious that its is not a part of the function body and its clear that it can only operate on the whole function. 		\item Because the expression is prefixed with @ is stands out and is hard to miss (``in your face'', according to the PEP[Python Enhancement Proposals] :) ). 
    \item When more than one decorator is applied, each one is placed on a separate line in an easy to read way.
    \end{itemize}
\end{frame}

%%%%%%%%%%%%%%%%%%%%%%%%%%%%%%%%%%%%%%%%%%%%%%%%%%%%%%%%%%%%%%%%%%%%%%%%%%%%%%%%%%%
\begin{frame}[fragile]\frametitle{Decorators Intro}
    \begin{itemize}
    \item The only requirement on decorators is that they can be called with a single argument. 
    \item This means that decorators can be implemented as normal functions, or as classes with a \_\_call\_\_ method, or in theory, even as lambda functions.
    \end{itemize}
\end{frame}

%%%%%%%%%%%%%%%%%%%%%%%%%%%%%%%%%%%%%%%%%%%%%%%%%%%%%%%%%%%%%%%%%%%%%%%%%%%%%%%%%%%
\begin{frame}[fragile]\frametitle{Decorators implemented as classes and as functions}
    \begin{itemize}
    \item The decorator expression (the part after @) can be either just a name, or a call. 
    \item The bare-name approach is nice (less to type, looks cleaner, etc.), but is only possible when no arguments are needed to customise the decorator. 
    \end{itemize}
            \begin{lstlisting}
def simple_decorator(function):
  print("doing decoration")
  return function
  
@simple_decorator
def function():
  print("inside function")

doing decoration
>>>function()
inside function
\end{lstlisting}
\end{frame}

%%%%%%%%%%%%%%%%%%%%%%%%%%%%%%%%%%%%%%%%%%%%%%%%%%%%%%%%%%%%%%%%%%%%%%%%%%%%%%%%%%%
\begin{frame}[fragile]\frametitle{Decorators implemented as classes and as functions}
            \begin{lstlisting}
def decorator_with_arguments(arg):
  print("defining the decorator")
  def _decorator(function):
      # in this inner function, arg is available too
      print("doing decoration, %r" % arg)
      return function
  return _decorator
@decorator_with_arguments("abc")
def function():
  print("inside function")

defining the decorator
doing decoration, `abc'
function()
inside function
\end{lstlisting}
\end{frame}

%%%%%%%%%%%%%%%%%%%%%%%%%%%%%%%%%%%%%%%%%%%%%%%%%%%%%%%%%%%%%%%%%%%%%%%%%%%%%%%%%%%
\begin{frame}[fragile]\frametitle{Decorators Example}
One of the most used decorators in Python is the login\_required() decorator, which ensures that a user is logged in/properly authenticated before s/he can access a specific route (/secret, in this case):
        \begin{lstlisting}
from functools import wraps
from flask import g, request, redirect, url_for

def login_required(f):
    @wraps(f)
    def decorated_function(*args, **kwargs):
        if g.user is None:
            return redirect(url_for('login', next=request.url))
        return f(*args, **kwargs)
    return decorated_function

@app.route('/secret')
@login_required
def secret():
    pass
\end{lstlisting}
\end{frame}


%%%%%%%%%%%%%%%%%%%%%%%%%%%%%%%%%%%%%%%%%%%%%%%%%%%%%%%%%%%%%%%%%%%%%%%%%%%%%%%%%%%
\begin{frame}[fragile]\frametitle{Decorators Example}
Did you notice that the function gets passed to the functools.wraps() decorator? This simply preserves the metadata of the wrapped function.
Let's look at one last use case. Take a quick look at the following Flask route handler:
        \begin{lstlisting}
@app.route('/grade', methods=['POST'])
def update_grade():
    json_data = request.get_json()
    if 'student_id' not in json_data:
        abort(400)
    # update database
    return "success!"
\end{lstlisting}
Here we ensure that the key student\_id is part of the request. Although this validation works it really does not belong in the function itself. Plus, perhaps there are other routes that use the exact same validation
\end{frame}

