%%%%%%%%%%%%%%%%%%%%%%%%%%%%%%%%%%%%%%%%%%%%%%%%%%%%%%%%%%%%%%%%%%%%%%%%%%%%%%%%%%
\begin{frame}[fragile]\frametitle{}

\begin{center}
{\Large Language Model}
\end{center}
\end{frame}


 %%%%%%%%%%%%%%%%%%%%%%%%%%%%%%%%%%%%%%%%%%%%%%%%%%%%%%%%%%%%%%%%%%%%%%%%%%%%%%%%%%
\begin{frame}[fragile]
  \frametitle{Language modeling}
  \begin{itemize}
  \item What's the probability of the next word?
  \item Can it be ``house'' or ``rat'' or ``malt''. Whats probability of each being the next?
    \item $P(house| \text{This is the }) = ?$
	\item You have read many books, seen texts, whats the likelihood of each, based on the reading?
	\item Lets train machine to do this, using a toy corpus.
  	  \end{itemize}
  	  
(Ref: https://www.coursera.org/learn/language-processing/lecture/IdJFl/count-n-gram-language-models)  	  
 \end{frame} 

 %%%%%%%%%%%%%%%%%%%%%%%%%%%%%%%%%%%%%%%%%%%%%%%%%%%%%%%%%%%%%%%%%%%%%%%%%%%%%%%%%%
\begin{frame}[fragile]
  \frametitle{Toy corpus}
  \begin{lstlisting}
This is the house that Jack built.
This is the malt
That lay in the house that Jack built.
This is the rat,
That ate the malt
That lay in the house that Jack built.
This is the cat,
That killed the rat,
That ate the malt
That lay in the house that Jack built.
  	  \end{lstlisting}
  	  
 Now, $P(house| \text{This is the }) = ?$
 
 \end{frame} 

 %%%%%%%%%%%%%%%%%%%%%%%%%%%%%%%%%%%%%%%%%%%%%%%%%%%%%%%%%%%%%%%%%%%%%%%%%%%%%%%%%%
\begin{frame}[fragile]
  \frametitle{Toy corpus}
  \begin{itemize}
  \item ``This is the'' comes 4 times.
  \item Each time with different next-word: house, malt, rat, cat.
  \item So $P(house| \text{This is the }) = 1/4$ and for other 3 as well.
  \item Another example $P(jack|that) = 4/10 = 0.4$
  \item We have seen such things at many places.

  	  \end{itemize}
  	  
 
 \end{frame} 

 %%%%%%%%%%%%%%%%%%%%%%%%%%%%%%%%%%%%%%%%%%%%%%%%%%%%%%%%%%%%%%%%%%%%%%%%%%%%%%%%%%
\begin{frame}[fragile]
  \frametitle{Where do we need Language Model?}
  \begin{itemize}
  \item Predictive words while typing on mobile.
  \item Spelling correction
  \item Machine translation
  \item Speech recognition
  \item Handwriting recognition
  	  \end{itemize}
  	  
 
 \end{frame} 


 %%%%%%%%%%%%%%%%%%%%%%%%%%%%%%%%%%%%%%%%%%%%%%%%%%%%%%%%%%%%%%%%%%%%%%%%%%%%%%%%%%
\begin{frame}[fragile]
  \frametitle{Mathematically}
  \begin{itemize}
  \item Predict probability of a sequence of words: $w = (w_1w_2w_3 \ldots w_k)$ 
  \item Chain rule: Take first word's probability. Probability of next is based on all the previous ones till that time: $p(w) = p(w_1 )p(w_2 | w_1 ) \ldots  p(w_k | w_1  \ldots w_{k-1} )$
  \item The last term is too long, complicated.
  \item Rescue: Markov says, don't look too much into history!!! May be last few(n) terms.
  	  \end{itemize}
  	  
 
 \end{frame} 

 %%%%%%%%%%%%%%%%%%%%%%%%%%%%%%%%%%%%%%%%%%%%%%%%%%%%%%%%%%%%%%%%%%%%%%%%%%%%%%%%%%
\begin{frame}[fragile]
  \frametitle{Markov}
$p(w) = p(w_1 )p(w_2 | w_1 ) \ldots  p(w_k | w_{k-1} )$

Toy corpus: 
  \begin{lstlisting}
This is the malt
That lay in the house that Jack built.
 \end{lstlisting}
  	  
$p(\text{this is the house}) = p(this) p(is| this) p(the| is) p(house| the) = 1/12 x 1 x 1 x 1/2 = 1/24$
 
 Problems?
 \end{frame} 

 %%%%%%%%%%%%%%%%%%%%%%%%%%%%%%%%%%%%%%%%%%%%%%%%%%%%%%%%%%%%%%%%%%%%%%%%%%%%%%%%%%
\begin{frame}[fragile]
  \frametitle{Problems}
  \begin{itemize}
  \item First word (start) is ``This or That'' in our toy corpus, and it can never be ``malt'' or ``cat''.
  \item We should modify the first one with a fake ``start'' token. $p(w_1|start)$ that 1/2 as there are two candidates ``This or That'
  \item Sentences are of varying lengths. Are this probabilities normalized then?
  \item All unigrams probabilities will sum up to 1. Similarly for bi-grams, trigrams, etc separately.
  \item But we want, total distribution of all possible sequences.
  \item Add fake token at end as well $p(end|w_k)$
  \item That makes all possible combinations FINITE and thus sum of probabilities will be 1.
  	  \end{itemize}
  	  
 \end{frame} 

 %%%%%%%%%%%%%%%%%%%%%%%%%%%%%%%%%%%%%%%%%%%%%%%%%%%%%%%%%%%%%%%%%%%%%%%%%%%%%%%%%%
\begin{frame}[fragile]
  \frametitle{Training of Language Models}
  \begin{itemize}
  \item  Sentence is $w = (w_1w_2w_3 \ldots w_k)$
  \item In Bigram language model, next word depnds on previus word: $p(w) = \Pi  p(w_i | w_{i-1}$
    \item In N-gram language model, next word depnds on past many words: $p(w) = \Pi  p(w_i | w^{i-1}_{i - n + 1})$
    \item We can estimate these probabilities by (a crude way) counting previous occurances.
    \item Better/rigourous way is Log Likelihood maximizatoin.
  	  \end{itemize}
  	  
 \end{frame} 
 
  %%%%%%%%%%%%%%%%%%%%%%%%%%%%%%%%%%%%%%%%%%%%%%%%%%%%%%%%%%%%%%%%%%%%%%%%%%%%%%%%%%
\begin{frame}[fragile]
  \frametitle{Log-likelihodd Maximization}
  \begin{itemize}
  \item  $\log p (w_{train}) = \sum \log p (w_i | w^{i-1}_{i-n+1}) \rightarrow max $
  \item Log is taken as the product becomes addition, easier to optimize.
  \item Here, $ p (w_i | w^{i-1}_{i-n+1}) = \frac{c(w^i_{i-n+1})}{\sum c(w^i_{i-n+1})} = \frac{c(w^i_{i-n+1})}{c(w^{i-1}_{i-n+1})} $
  \item Counts normalzied by total counts.
  	  \end{itemize}
  	  
 \end{frame} 
 
   %%%%%%%%%%%%%%%%%%%%%%%%%%%%%%%%%%%%%%%%%%%%%%%%%%%%%%%%%%%%%%%%%%%%%%%%%%%%%%%%%%
\begin{frame}[fragile]
  \frametitle{Evaluation of Model on Test data}
  \begin{itemize}
  \item  Likelihood: $L =  p (w_{test}) = \sum p (w_i | w^{i-1}_{i-n+1}) $
  \item Perplexisty $P = p (w_{test})^{- 1/N}$
  \item Somewhat related to Entropy. Lower the better. More the likelihood. Good Predictions.
  	  \end{itemize}
  	  
 \end{frame} 
 
    %%%%%%%%%%%%%%%%%%%%%%%%%%%%%%%%%%%%%%%%%%%%%%%%%%%%%%%%%%%%%%%%%%%%%%%%%%%%%%%%%%
\begin{frame}[fragile]
  \frametitle{Example I}
  \begin{itemize}
  \item Toy train corpus:
This is the house that Jack built.
\item Toy test corpus:
This is the \textbf{malt}.
\item What's the perplexity of the Bigram LM?
  	  \end{itemize}
  	  
 \end{frame} 

 
    %%%%%%%%%%%%%%%%%%%%%%%%%%%%%%%%%%%%%%%%%%%%%%%%%%%%%%%%%%%%%%%%%%%%%%%%%%%%%%%%%%
\begin{frame}[fragile]
  \frametitle{Example I}
  \begin{itemize}
  \item Start computing probabilities first.
  \item $p(malt | the) = \frac{c(the malt)}{c(the)} = 0$
  \item $p(w_{test} ) = 0$ so $P = \infty$ beacuse ``malt'' is there in test but was not in train. Not good $P$.
  \item How to fix?
  	  \end{itemize}
  	  
 \end{frame} 
 
     %%%%%%%%%%%%%%%%%%%%%%%%%%%%%%%%%%%%%%%%%%%%%%%%%%%%%%%%%%%%%%%%%%%%%%%%%%%%%%%%%%
\begin{frame}[fragile]
  \frametitle{Example I}
  Simple idea:
  \begin{itemize}
  \item Build a vocabulary (e.g. by word frequencies)
  \item Substitute Out-of-vocabulary words by $\text{<UNK>}$  (both in train and test!)
  \item Compute counts as usual for all tokens.
  \item Fixed.
  	  \end{itemize}
  	  
 \end{frame} 


    %%%%%%%%%%%%%%%%%%%%%%%%%%%%%%%%%%%%%%%%%%%%%%%%%%%%%%%%%%%%%%%%%%%%%%%%%%%%%%%%%%
\begin{frame}[fragile]
  \frametitle{Example II}
  Lets say that we dont have any Out-of-vocabulary words
  \begin{itemize}
  \item Toy train corpus:
This is the house that Jack built.
  \item Toy test corpus:
This is \textbf{Jack}.
  \item What’s the perplexity of the Bigram LM?
  \item  $p(Jack | is) = \frac{c(is Jack)}{c(is)} = 0$
  \item $p(w_{test}) = 0$, So $P = \infty$ beacuse ``is Jack'' is not in train. Not good $P$.
  \item How to fix? Smoothing technique.
  	  \end{itemize}
  	  
 \end{frame} 
 
     %%%%%%%%%%%%%%%%%%%%%%%%%%%%%%%%%%%%%%%%%%%%%%%%%%%%%%%%%%%%%%%%%%%%%%%%%%%%%%%%%%
\begin{frame}[fragile]
  \frametitle{Laplacian Smoothing}

  \begin{itemize}
  \item Having 0 probability is not good. 
  	\item We can not just assign it to 1 also, as sum of all such probabilities need to be 1.
  	\item TricK: Add 1 to numerator and all items in denominator.
  	$\hat{p}(w_i | w^{i-1}_{i-n+1}) = \frac{c(w^i_{i-n+1}) + 1}{c(w^i_{i-n+1}) + V}$
  	\item Its called ``add-one smoothing''
  	\item Instead of 1 is you add $k$, its called ``add-k smoothing''.
  	  \end{itemize}
  	  
 \end{frame} 


