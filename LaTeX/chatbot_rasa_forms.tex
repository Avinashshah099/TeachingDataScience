%%%%%%%%%%%%%%%%%%%%%%%%%%%%%%%%%%%%%%%%%%%%%%%%%%%%%%%%%%%%%%%%%%%%%%%%%%%%%%%%%%
\begin{frame}[fragile]\frametitle{}
\begin{center}
{\Large Forms}

{\tiny (Ref: https://rasa.com/docs/rasa/core/forms/)}
\end{center}
\end{frame}

%%%%%%%%%%%%%%%%%%%%%%%%%%%%%%%%%%%%%%%%%%%%%%%%%%%%%%%%%%%
 \begin{frame}[fragile]\frametitle{What is a Form?}
\begin{itemize}
\item Most common pattern in chatbot, is to collect a few pieces of information from a user in order to do something (book a restaurant, call an API, search a database, etc.). 
\item This is also called slot filling.
\item If you need to collect multiple pieces of information in a row, use FormAction. 
\item This is a single action which contains the logic to loop over the required slots and ask the user for this information. 
\end{itemize}
\end{frame}


%%%%%%%%%%%%%%%%%%%%%%%%%%%%%%%%%%%%%%%%%%%%%%%%%%%%%%%%%%%
 \begin{frame}[fragile]\frametitle{How to use Form?}

\begin{itemize}
\item Add form name to the domain file
\begin{lstlisting}
forms:
  - restaurant_form
actions:
  ...
\end{lstlisting}
\item Include the FormPolicy in your policy configuration file.
\begin{lstlisting}
policies:
  - name: "FormPolicy"
\end{lstlisting}
\end{itemize}
\end{frame}

%%%%%%%%%%%%%%%%%%%%%%%%%%%%%%%%%%%%%%%%%%%%%%%%%%%%%%%%%%%
 \begin{frame}[fragile]\frametitle{Form Basics}
\begin{itemize}
\item Using a FormAction, you can describe all of the happy paths with a single story.
\begin{lstlisting}
## happy path
* request_restaurant
    - restaurant_form
    - form{"name": "restaurant_form"}
    - form{"name": null}
\end{lstlisting}
\item The User intent is ``request\_restaurant''
\item The the form action ``restaurant\_form''
\item With form{"name": "restaurant\_form"} the form is activated
\item With form{"name": null} the form is deactivated again
\end{itemize}
\end{frame}

%%%%%%%%%%%%%%%%%%%%%%%%%%%%%%%%%%%%%%%%%%%%%%%%%%%%%%%%%%%
 \begin{frame}[fragile]\frametitle{Form Action}
\begin{itemize}
\item The FormAction will only request slots which haven’t already been set. 
\item If a user starts the conversation with : ``I’d like a vegetarian Chinese restaurant for 8 people''
\item then they won’t be asked about the ``cuisine'' and ``num\_people'' slots.
\end{itemize}
\end{frame}

%%%%%%%%%%%%%%%%%%%%%%%%%%%%%%%%%%%%%%%%%%%%%%%%%%%%%%%%%%%
 \begin{frame}[fragile]\frametitle{restaurant\_form}
In actions.py for class restaurant\_form define following methods:
\begin{itemize}
\item name: the name of this action
\item required\_slots: a list of slots that need to be filled for the submit method to work.
\item submit: what to do at the end of the form, when all the slots have been filled.
\end{itemize}
\begin{lstlisting}
def name(self) -> Text:
    """Unique identifier of the form"""

    return "restaurant_form"
\end{lstlisting}
\end{frame}

%%%%%%%%%%%%%%%%%%%%%%%%%%%%%%%%%%%%%%%%%%%%%%%%%%%%%%%%%%%
 \begin{frame}[fragile]\frametitle{restaurant\_form}
actions.py
\begin{lstlisting}
@staticmethod
def required_slots(tracker: Tracker) -> List[Text]:
    """A list of required slots that the form has to fill"""

    return ["cuisine", "num_people", "outdoor_seating", "preferences", "feedback"]
def submit(self,dispatcher: CollectingDispatcher,
    tracker: Tracker,domain: Dict[Text, Any],) -> List[Dict]:
    """Define what the form has to do
        after all required slots are filled"""

    # utter submit template
    dispatcher.utter_template("utter_submit", tracker)
    return []
\end{lstlisting}
\end{frame}

%%%%%%%%%%%%%%%%%%%%%%%%%%%%%%%%%%%%%%%%%%%%%%%%%%%%%%%%%%%
 \begin{frame}[fragile]\frametitle{Form Action}
\begin{itemize}
\item Every time the form action gets called, it will ask the user for the next slot in required\_slots which is not already set. 
\item It does this by looking for a template called ``utter\_ask\_{slot\_name}'', 
\item So you need to define these in your domain file for each required slot.
\item Once all the slots are filled, the submit() method is called, where you can to query a restaurant API.
\end{itemize}
\end{frame}