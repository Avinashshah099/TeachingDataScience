%%%%%%%%%%%%%%%%%%%%%%%%%%%%%%%%%%%%%%%%%%%%%%%%%%%%%%%%%%%%%%%%%%%%%%%%%%%%%%%%%%
\begin{frame}[fragile]\frametitle{}
\begin{center}
{\Large Loops}
\end{center}
\end{frame}


%%%%%%%%%%%%%%%%%%%%%%%%%%%%%%%%%%%%%%%%%%%%%%%%%%%%%%%%%%%%%%%%%%%%%%%%%%%%%%%%%%%
\begin{frame}[fragile]\frametitle{Looping}
\begin{itemize}
\item Computers are often used to automate repetitive tasks. 
\item  Repeating identical or similar tasks without making errors is something that computers do well and people
do poorly. 
\end{itemize}
\begin{lstlisting}
n = 5
while n > 0:
	print(n)
	n = n - 1
print('Blastoff!')
\end{lstlisting}
\end{frame}



%%%%%%%%%%%%%%%%%%%%%%%%%%%%%%%%%%%%%%%%%%%%%%%%%%%%%%%%%%%%%%%%%%%%%%%%%%%%%%%%%%%
\begin{frame}[fragile]
  \frametitle{\texttt{for}-loops}
    With the  \texttt{for} statement, you can loop over the items of
    a sequence:
\begin{lstlisting}
for i in range(0, 4):
  # loop block
  print (i*i)
\end{lstlisting}

  
  To break out of a \texttt{for} loop, use the \texttt{break}
  statement.

  
  To jump to the next iteration of a \texttt{for} loop, use the
  \texttt{continue} statement.
\end{frame}

%%%%%%%%%%%%%%%%%%%%%%%%%%%%%%%%%%%%%%%%%%%%%%%%%%%%%%%%%%%%%%%%%%%%%%%%%%%%%%%%%%%
\begin{frame}[fragile]\frametitle{\texttt{for}}
  The \texttt{for} statement can be used to loop over elements in \emph{any sequence}.


\begin{lstlisting}
>>> for val in [1,2,3]:
...   print(val)
1
2
3
\end{lstlisting}
\end{frame}

%%%%%%%%%%%%%%%%%%%%%%%%%%%%%%%%%%%%%%%%%%%%%%%%%%%%%%%%%%%%%%%%%%%%%%%%%%%%%%%%%%%
\begin{frame}[fragile]\frametitle{Looping over range of numbers}
Two ways:
\begin{lstlisting}
for i in [0, 1, 2, 3, 4, 5]:
    print i**2

for i in range(6):
    print i**2
\end{lstlisting}
Better:
\begin{lstlisting}
for i in xrange(6):
    print i**2
\end{lstlisting}
xrange creates an iterator over the range producing the values one at a time. This approach is much more memory efficient than range. xrange was renamed to range in python 3

\tiny{(Ref: Transforming Code into Beautiful, Idiomatic Python -  Raymond Hettinger)}
\end{frame}

%%%%%%%%%%%%%%%%%%%%%%%%%%%%%%%%%%%%%%%%%%%%%%%%%%%%%%%%%%%%%%%%%%%%%%%%%%%%%%%%%%%
\begin{frame}[fragile]\frametitle{Looping over a collection}
``C'' way:
\begin{lstlisting}
colors = ['red', 'green', 'blue', 'yellow']

for i in range(len(colors)):
    print(colors[i])
\end{lstlisting}
Better:
\begin{lstlisting}
for color in colors:
    print(color)
\end{lstlisting}

\tiny{(Ref: Transforming Code into Beautiful, Idiomatic Python -  Raymond Hettinger)}
\end{frame}

%%%%%%%%%%%%%%%%%%%%%%%%%%%%%%%%%%%%%%%%%%%%%%%%%%%%%%%%%%%%%%%%%%%%%%%%%%%%%%%%%%%
\begin{frame}[fragile]\frametitle{\texttt{for}}
  The \texttt{for} statement can be used to loop over elements in \emph{any sequence}.

\begin{lstlisting}
>>> for val in {'UZH'}:
...   print(val)
'U'
'Z'
'H'
\end{lstlisting}

\end{frame}

%%%%%%%%%%%%%%%%%%%%%%%%%%%%%%%%%%%%%%%%%%%%%%%%%%%%%%%%%%%%%%%%%%%%%%%%%%%%%%%%%%%
\begin{frame}[fragile]\frametitle{Looping backwards}
``C'' way:
\begin{lstlisting}
colors = ['red', 'green', 'blue', 'yellow']

for i in range(len(colors)-1, -1, -1):
    print(colors[i])
\end{lstlisting}
Better:
\begin{lstlisting}
for color in reversed(colors):
    print(color)
\end{lstlisting}

\tiny{(Ref: Transforming Code into Beautiful, Idiomatic Python -  Raymond Hettinger)}
\end{frame}






%%%%%%%%%%%%%%%%%%%%%%%%%%%%%%%%%%%%%%%%%%%%%%%%%%%%%%%%%%%%%%%%%%%%%%%%%%%%%%%%%%%
\begin{frame}[fragile]\frametitle{Sorted}

  If you want to loop over a \textit{sorted} sequence you can use the
  function \texttt{sorted()} :

  \begin{lstlisting}
>>> for val in sorted([1,3,4,2]):
...  print(val)
1
2
3
4
  \end{lstlisting}

and to loop over a sequence in \textit{inverted} order you can use the
\texttt{reversed()} function:

\begin{lstlisting}
>>> for val in reversed([1,3,4,2]):
...     print(val)
2
4
3
1
\end{lstlisting}

\end{frame}

%%%%%%%%%%%%%%%%%%%%%%%%%%%%%%%%%%%%%%%%%%%%%%%%%%%%%%%%%%%%%%%%%%%%%%%%%%%%%%%%%%%
\begin{frame}[fragile]\frametitle{Looping over a collection and indices}
``C'' way:
\begin{lstlisting}
colors = ['red', 'green', 'blue', 'yellow']

for i in range(len(colors)):
    print(i, '--->', colors[i])
\end{lstlisting}
Better:
\begin{lstlisting}
for i, color in enumerate(colors):
    print(i, '--->', color)
\end{lstlisting}

It's fast and beautiful and saves you from tracking the individual indices and incrementing them.

Whenever you find yourself manipulating indices [in a collection], you're probably doing it wrong


\tiny{(Ref: Transforming Code into Beautiful, Idiomatic Python -  Raymond Hettinger)}
\end{frame}

%%%%%%%%%%%%%%%%%%%%%%%%%%%%%%%%%%%%%%%%%%%%%%%%%%%%%%%%%%%%%%%%%%%%%%%%%%%%%%%%%%%
\begin{frame}[fragile]\frametitle{Looping over two collections}
``C'' way:
\begin{lstlisting}
names = ['raymond', 'rachel', 'matthew']
colors = ['red', 'green', 'blue', 'yellow']

n = min(len(names), len(colors))
for i in range(n):
    print(names[i], '--->', colors[i])

for name, color in zip(names, colors):
    print(name, '--->', color)
\end{lstlisting}
Better:
\begin{lstlisting}
for name, color in zip(names, colors):
    print(name, '--->', color)
\end{lstlisting}

zip creates a new list in memory and takes more memory. izip is more efficient than zip. Note: in python 3 izip was renamed to zip and promoted to a builtin replacing the old zip.


\tiny{(Ref: Transforming Code into Beautiful, Idiomatic Python -  Raymond Hettinger)}
\end{frame}


%%%%%%%%%%%%%%%%%%%%%%%%%%%%%%%%%%%%%%%%%%%%%%%%%%%%%%%%%%%%%%%%%%%%%%%%%%%%%%%%%%%
\begin{frame}[fragile]\frametitle{Looping}
  Conditional looping uses the \lstinline{while} statement:
\begin{lstlisting}
while expr:
  # indented block
 else:
   # executed at natural end of the loop
\end{lstlisting}

\begin{itemize}
\item To break loop, use the \lstinline{break} statement.
\item Use \lstinline{continue} anywhere inside to jump back to the \lstinline{while}.
\item If a loop is exited via a \lstinline{break} statement, the \lstinline{else} is not executed.
\item \lstinline{else} is optional.
\end{itemize}

\end{frame}


%%%%%%%%%%%%%%%%%%%%%%%%%%%%%%%%%%%%%%%%%%%%%%%%%%%%%%%%%%%%%%%%%%%%%%%%%%%%%%%%%%%
\begin{frame}[fragile]\frametitle{Looping Exercises}
\begin{itemize}
\item Write a Python program to count the number 4 in a given list. 
\item Write a Python program to create a histogram from a given list of integers.
\end{itemize}
\end{frame}



%%%%%%%%%%%%%%%%%%%%%%%%%%%%%%%%%%%%%%%%%%%%%%%%%%%%%%%%%%%%%%%%%%%%%%%%%%%%%%%%%%%
\begin{frame}[fragile]\frametitle{Looping}
To find the largest value in a list or sequence, we construct the following loop:
\begin{lstlisting}
largest = None
print('Before:', largest)
for itervar in [3, 41, 12, 9, 74, 15]:
	if largest is None or itervar > largest :
		largest = itervar
	print('Loop:', itervar, largest)
print('Largest:', largest)
\end{lstlisting}
Finding smallest, the code is very similar with one small change:
\begin{lstlisting}
smallest = None
print('Before:', smallest)
for itervar in [3, 41, 12, 9, 74, 15]:
	if smallest is None or itervar < smallest:
		smallest = itervar
	print('Loop:', itervar, smallest)
print('Smallest:', smallest)
\end{lstlisting}
\end{frame}

%%%%%%%%%%%%%%%%%%%%%%%%%%%%%%%%%%%%%%%%%%%%%%%%%%%%%%%%%%%%%%%%%%%%%%%%%%%%%%%%%%%
\begin{frame}[fragile]\frametitle{Maximum and minimum loops}
Automatic Stop:
\begin{lstlisting}
n = 10
while n:
	print(n)
	n = n - 1
print('Done!')
\end{lstlisting}
Forced stop:
\begin{lstlisting}
while True:
	line = input('> ')
	if line == 'done':
		break
	print(line)
print('Done!')
\end{lstlisting}
\end{frame}

%%%%%%%%%%%%%%%%%%%%%%%%%%%%%%%%%%%%%%%%%%%%%%%%%%%%%%%%%%%%%%%%%%%%%%%%%%%%%%%%%%%
\begin{frame}[fragile]\frametitle{Guess the number}

Output looks like this:
\begin{lstlisting}
Hello! What is your name?
Albert
Well, Albert, I am thinking of a number between 1 and 20.
Take a guess.
10
Your guess is too high.
Take a guess.
2
Your guess is too low.
Take a guess.
4
Good job, Albert! You guessed my number in 3 guesses!
\end{lstlisting}
Write the program to do this. Use `random' module to get the secret number.
\end{frame}


%%%%%%%%%%%%%%%%%%%%%%%%%%%%%%%%%%%%%%%%%%%%%%%%%%%%%%%%%%%%%%%%%%%%%%%%%%%%%%%%%%%
\begin{frame}[fragile]\frametitle{Guess the number}
\begin{lstlisting}
import random
guessesTaken = 0
number = random.randint(1, 20)
print('Think a number between 1 and 20.')
while guessesTaken < 6:
     print('Take a guess.') 
     guess = input()
     guess = int(guess)
     guessesTaken = guessesTaken + 1
     if guess < number:
         print('Your guess is too low.') 
     if guess > number:
        print('Your guess is too high.')
     if guess == number:
         break
if guess == number:
     guessesTaken = str(guessesTaken)
     print(`You guessed in ' + guessesTaken + ' guesses!')
if guess != number:
     number = str(number)
     print('Nope. The number I was thinking of was ' + number)
\end{lstlisting}
\end{frame}

%%%%%%%%%%%%%%%%%%%%%%%%%%%%%%%%%%%%%%%%%%%%%%%%%%%%%%%%%%%%%%%%%%%%%%%%%%%%%%%%%%%
\begin{frame}[fragile]\frametitle{Quiz: Matrix Multiplication}
Given X and Y write code to do matrix multiplication
\begin{lstlisting}
X = [[12,7,3],
    [4 ,5,6],
    [7 ,8,9]]

Y = [[5,8,1,2],
    [6,7,3,0],
    [4,5,9,1]]

result = [[0,0,0,0],
         [0,0,0,0],
         [0,0,0,0]]
\end{lstlisting}

\end{frame}

%%%%%%%%%%%%%%%%%%%%%%%%%%%%%%%%%%%%%%%%%%%%%%%%%%%%%%%%%%%%%%%%%%%%%%%%%%%%%%%%%%%
\begin{frame}[fragile]\frametitle{Solution : Matrix Multiplication}

\begin{lstlisting}
for i in range(len(X)):
    for j in range(len(Y[0])):
        for k in range(len(Y)):
            result[i][j] += X[i][k] * Y[k][j]

for r in result:
    print(r)
\end{lstlisting}

\end{frame}
% %%%%%%%%%%%%%%%%%%%%%%%%%%%%%%%%%%%%%%%%%%%%%%%%%%%%%%%%%%%%%%%%%%%%%%%%%%%%%%%%%%%
% \begin{frame}[fragile]\frametitle{Unpacking}
% The  \lstinline{for}:  statement can also do unpacking. Example:
% \begin{lstlisting}
% items = ['apple', 'banana', 'cherry', 'date']
% for idx, item in enumerate(items):
	% print('{}. {}'.format(idx, item))
	
% 0.  apple
% 1.  banana
% 2.  cherry
% 3.  date
% \end{lstlisting}

% \begin{itemize}
% \item \lstinline|[f(x) for x in iterable]|
% \item \lstinline|[f(x) for x in iterable  if t(x)]|
% \end{itemize}

% \end{frame}


% %%%%%%%%%%%%%%%%%%%%%%%%%%%%%%%%%%%%%%%%%%%%%%%%%%%%%%%%%%%%%%%%%%%%%%%%%%%%%%%%%%%
% \begin{frame}[fragile]\frametitle{Generator}
% Generator expressions -- A generator expression looks similar to a list comprehension, except that it is surrounded by parentheses rather than square
% brackets. Example:
% \begin{lstlisting}
% items = ['apple', 'banana', 'cherry', 'date']
% gen1 = (item.upper() for item in items)
% for x in gen1:
	% print('x: {}'.format(x))
	
% x: APPLE
% x: BANANA
% x: CHERRY
% x: DATE
% \end{lstlisting}
% \end{frame}
