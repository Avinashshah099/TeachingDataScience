%%%%%%%%%%%%%%%%%%%%%%%%%%%%%%%%%%%%%%%%%%%%%%%%%%%%%%%%%%%%%%%%%%%%%%%%%%%%%%%%%%
\begin{frame}[fragile]\frametitle{}
\begin{center}
{\Large Fallback Policy}

{\tiny (Ref: https://rasa.com/docs/rasa/core/fallback-actions/ )}
\end{center}
\end{frame}

%%%%%%%%%%%%%%%%%%%%%%%%%%%%%%%%%%%%%%%%%%%%%%%%%%%%%%%%%%%
 \begin{frame}[fragile]\frametitle{Why Fallback?}
\begin{itemize}
\item At certain point, due to insufficient, incorrect user response, the dialog management system is not able to decide the NEXT step.
\item Typically, you give\-up saying that ``Sorry, I didn’t understand that''.
\item There are various ways to handle such cases called ``Fallbacks''.
\end{itemize}
\end{frame}

%%%%%%%%%%%%%%%%%%%%%%%%%%%%%%%%%%%%%%%%%%%%%%%%%%%%%%%%%%%
 \begin{frame}[fragile]\frametitle{How: Fallback?}
\begin{itemize}
\item FallbackPolicy has one fallback action
\item Will be executed if the intent recognition has a confidence below nlu\_threshold or
\item If none of the dialogue policies predict an action with confidence higher than core\_threshold
\end{itemize}
\end{frame}

%%%%%%%%%%%%%%%%%%%%%%%%%%%%%%%%%%%%%%%%%%%%%%%%%%%%%%%%%%%
 \begin{frame}[fragile]\frametitle{Configuring Fallback}
\begin{lstlisting}
policies:
  - name: "FallbackPolicy"
    nlu_threshold: 0.4
    core_threshold: 0.3
    fallback_action_name: "action_default_fallback"
\end{lstlisting}
\end{frame}

%%%%%%%%%%%%%%%%%%%%%%%%%%%%%%%%%%%%%%%%%%%%%%%%%%%%%%%%%%%
 \begin{frame}[fragile]\frametitle{Default Fallback}
\begin{itemize}
\item ``action\_default\_fallback'' is a default action
\item Rasa Core which sends the ``utter\_default'' template message to the user. 
\item Make sure to specify the ``utter\_default'' in your domain file. 
\item Will also revert back to the state of the conversation before the user message that caused the fallback, 
\item So that it will not influence the prediction of future actions. 
\end{itemize}
\end{frame}

%%%%%%%%%%%%%%%%%%%%%%%%%%%%%%%%%%%%%%%%%%%%%%%%%%%%%%%%%%%
 \begin{frame}[fragile]\frametitle{Custom Fallback}
\begin{lstlisting}
policies:
  - name: "FallbackPolicy"
    nlu_threshold: 0.4
    core_threshold: 0.3
    fallback_action_name: "my_fallback_action"
\end{lstlisting}

Derive your action from \lstinline|ActionDefaultFallback|
\end{frame}

%%%%%%%%%%%%%%%%%%%%%%%%%%%%%%%%%%%%%%%%%%%%%%%%%%%%%%%%%%%
 \begin{frame}[fragile]\frametitle{Custom Fallback Story}
 If you have a specific intent, let's say it's called ``out\_of\_scope'', that should always trigger the fallback action, add this as a story:
\begin{lstlisting}
## fallback story
* out_of_scope
  - action_default_fallback
\end{lstlisting}

Derive your action from \lstinline|ActionDefaultFallback|
\end{frame}

%%%%%%%%%%%%%%%%%%%%%%%%%%%%%%%%%%%%%%%%%%%%%%%%%%%%%%%%%%%
 \begin{frame}[fragile]\frametitle{Two-stage Fallback Policy}
\begin{itemize}
\item TwoStageFallbackPolicy handles low {\bf NLU} confidence in multiple stages by trying to disambiguate the user input 
\item Low {\bf core} confidence is handled in the same manner as the FallbackPolicy
\item If NLU has low confidence, then user is asked to affirm the intent (Default action: ``action\_default\_ask\_affirmation'')
\item If they affirm, the story continues as if the intent was classified with high confidence from the beginning.
\item If they deny, the user is asked to rephrase their message.
\end{itemize}
\end{frame}

%%%%%%%%%%%%%%%%%%%%%%%%%%%%%%%%%%%%%%%%%%%%%%%%%%%%%%%%%%%
 \begin{frame}[fragile]\frametitle{Two-stage Fallback Policy}
 Rephrasing (default action: ``action\_default\_ask\_rephrase''):
\begin{itemize}
\item If the classification of the rephrased intent was confident, the story continues as if the user had this intent from the beginning.
\item If the rephrased intent was not classified with high confidence, the user is asked to affirm the classified intent.
\end{itemize}
\end{frame}

%%%%%%%%%%%%%%%%%%%%%%%%%%%%%%%%%%%%%%%%%%%%%%%%%%%%%%%%%%%
 \begin{frame}[fragile]\frametitle{Two-stage Fallback Policy}
 Second affirmation (default action: ``action\_default\_ask\_affirmation''):
\begin{itemize}
\item If the user affirms the intent, the story continues as if the user had this intent from the beginning.
\item If the user denies, the original intent is classified as the specified ``deny\_suggestion\_intent\_name'', and an ultimate fallback action ``fallback\_nlu\_action\_name'' is triggered (e.g. a handoff to a human).
\end{itemize}
\end{frame}

%%%%%%%%%%%%%%%%%%%%%%%%%%%%%%%%%%%%%%%%%%%%%%%%%%%%%%%%%%%
 \begin{frame}[fragile]\frametitle{Two-stage Fallback Policy}
Rasa Core provides the default implementations of:
\begin{itemize}
\item ``action\_default\_ask\_affirmation''
\item ``action\_default\_ask\_rephrase'' utters the response template ``utter\_ask\_rephrase'' (be sure to specify this template in your domain file)
\item The implementation of both actions can be overwritten with custom actions
\end{itemize}
\end{frame}

%%%%%%%%%%%%%%%%%%%%%%%%%%%%%%%%%%%%%%%%%%%%%%%%%%%%%%%%%%%
 \begin{frame}[fragile]\frametitle{Two-stage Fallback Policy}
\begin{lstlisting}
policies:
  - name: TwoStageFallbackPolicy
    nlu_threshold: 0.3
    core_threshold: 0.3
    fallback_core_action_name: "action_default_fallback"
    fallback_nlu_action_name: "action_default_fallback"
    deny_suggestion_intent_name: "out_of_scope"
\end{lstlisting}

\end{frame}