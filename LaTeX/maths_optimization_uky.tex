%%%%%%%%%%%%%%%%%%%%%%%%%%%%%%%%%%%%%%%%%%%%%%%%%%%%%%%%%%%
 \begin{frame}[fragile] \frametitle{Notes from Avinash Sathaye}
https://http://www.ms.uky.edu/~sohum/ma162/fa\_09/lectures/
\end{frame}

\begin{frame} %2

  \frametitle{Inequalities. }
  We discuss inequalities in two or more variables.
                                               

 
  \begin{itemize}%[<+-| alert@+>]

\item An inequality in one variable looks like $2x+3\le 5$ and is solved by
  rearranging it so only the variable appears on the left hand side:
  $x\le 1$.   
   
\item This can also be done graphically thus:

Convert it to an equation and solve it. Thus:
$$2x+3=5 \mbox{ leads to } x=1.$$

\item On the number line,  plot the point $ x = 1$  
and notice that all points to the left of it satisfy the inequality 
and the ones on the right don't.
\end{itemize}

\end{frame}

%2


\begin{frame}%3
  \frametitle{Examples continued.}
  \begin{itemize}%[<+-| alert@+>]

\item The interval $(-\infty,1)$ on the number line looks like:
\centerline{\pict{1.5}{0.5}{interval1_lec7.jpg}}

We verify test values $x=0$ and $x=2$ to decide that this interval
consists of the solutions and the other part of the number line does
not.

\item The set of solutions is said to be {\bf the feasible set }of the
inequalities used.

\item If we similarly handle another inequality, say $3x+10\ge 4$, then
the solution to the associated equation $3x+10=4$ is $x=-2$ and the
interval $[-2,\infty)$ is deduced as before.

\centerline{\pict{1.5}{0.3}{interval2_lec7.jpg}}


\end{itemize}
\end{frame}

%3

\begin{frame}%4
  \frametitle{Examples continued.}
  \begin{itemize}%[<+-| alert@+>]
\item If we try to solve both $2x+3\le 5$ and $3x+10\ge 4$ together,
then we get the intersection of the two intervals. But this can be also
explaind thus.

\item We solve both associated equations plotting their solutions on the
number line. 
\centerline{\pict{1.5}{0.3}{interval3_lec7.jpg}}
 
\item By using test points on each interval, say $x=-3,0,2$ we pick up
the ones which satisfy all the inequalities. This gives the feasible set
$[-2,1]$.

\end{itemize}

\end{frame}

%4

\begin{frame}%5
  \frametitle{Inequalities in two variables.}
  \begin{itemize}%[<+-| alert@+>]
 
\item An inequality like $x+2y\le 4$
is the next topic. As before, we first convert it to the equation
$x+2y = 4$.

We note that this is a line and we know how to plot it.
It is not difficult to see that the plane is split into two halves
so that on one side of the line the inequality is true, while on the
other side it is not! 
Thus,  having plotted $x+2y = 4$, we see that at the origin $O(0,0)$ the
inequality is satisfied.

So, we choose as the feasible set the half plane containing the origin.

\begin{tabular}{ll}
\pict{1}{1}{region1_lec7.jpg} & 
\parbox[b]{2.9in}{\vspace{-0.8in}In the picture, only the first quadrant is shown,
since inequalities $x\ge 0,y\ge 0$ are typically going to be part
of our conditions.}\\
\end{tabular}
%\pause


\end{itemize}
\end{frame}

%5

\begin{frame}%6
  \frametitle{Two variables continued.}
  \begin{itemize}%[<+-| alert@+>]
 
\item
If we have more than one inequalities, then we solve them
separately and take the common part.
Here is the solution for
$$x\ge 0, y\ge 0, 3x+4y\le 12, x+2y\ge 2.$$                                

As before, the first two inequalities mean we only draw things in the
first quadrant.
%\pause
Here are the separate regions for the two inequalities followed by the
combined region.
\item
\begin{tabular}{ccc}
\parbox{1.5in}{$3x+4y\le 12$\\\pict{1}{1}{region2_lec7.jpg}} & 
\parbox{1.5in}{$x+2y\ge 2$\\\pict{1}{1}{region3_lec7.jpg}} &
\parbox{1.5in}{$Combined.$\\\pict{1}{1}{region4_lec7.jpg}}\\
\end{tabular}
%\pause


\end{itemize}
\end{frame}

%6

\begin{frame}%7
  \frametitle{Summary of The Graphical Method.}
  \begin{itemize}%[<+-| alert@+>]
\item
On a common graph paper, draw the equations corresponding to each inequality,
and mark the regions indicated by each inequality using a directional arrow.
In our course, the assumption always puts the region in the first
quadrant.
\item The directional arrow is usually decided by using a test point. For
any inequality at least one of the three points $(0,0), (1,0), (0,1)$ is always a
good point to use.
\item Take the common part of the plotted regions.
\item Calculate and list all the corner points. Make sure that the
chosen corner points actually satisfy all ineualities.
\item Some corner points, or even some whole lines may be lost, meaning
they do not have any points in the region.

\end{itemize}
\end{frame}

%7

\begin{frame}%8
 \frametitle{Summary Continued.}
  \begin{itemize}%[<+-| alert@+>]

\item
The aim of sketching and marking the corner points is to solve an
optimization problem for a linear function on the resulting region.
\item Review various examples in the section 3.2 where a practical
situation leads to a set of inequalities and a linear function to
be optimized (i.e. maximized or minimized). 
\item The first step is to clearly name the variables and write
down the inequalities and the function to be optimized.
\item
Then you sketch the region carefully, provided that you have
only two variables.
More variables are handled in the next chapter using the Simplex
algorithm.
\item 
Then you list all the corner points of the region in order and
evaluate the function at each of the corners. 
The optimum value of the function is among the values at
the corner points, with one exception, which we discuss next.

\end{itemize}
 
\end{frame}

%8

\begin{frame}%9
 \frametitle{Further Comments.}
  \begin{itemize}%[<+-| alert@+>]
\item If the region is not bounded, then the function may not
have a maximum or a minimum. This can be decided by checking
along the edges of the polygon running off to infinity.

\item {\bf Why does the graphical method work?}
We give a brief explanation below.
\item \mbl{\bf Parametric lines.}
Consider a line in the plane, say, $y = 3x+5$. 
It passes through a point $(1,8)$.
We want to study the line  near this point. 
So we  take a point on this line whose $x$-coordinate is $1+t$
and notice that its corresponding $y$-coordinate shall be 
$y=3(1+t)+5 = 8+3t$.
\item 
In fact, all points of this line can now be described
by the parametric equations:
$$x = 1+t,y = 8+3t.$$
This is called {\bf the parametric form }of the line.
It is useful to be able to calculate such a form 
near any convenient point!



\end{itemize}
 
\end{frame}

%9

\begin{frame}%10
  \frametitle{Explanation Continued.}
  \begin{itemize}%[<+-| alert@+>]
\item
Thus for the same line, we could also have started with a point
$(-2,-1)$ and concluded a different parametric form
$$x=-2+t, y=-1+3t.$$  
\item
\mbl{\bf Functions on Parametric lines.}
Consider a function, say $f(x,y)=3x+4y$.
We can analyze how it behaves on our parametric line by plugging
in the parametric form. Thus we have $f(x,y)=35+15t.$
\item This shows that as $t$ increases, so does the function value.
Remember that the parameter $t$ was the change in the $x$-coordinate
from the point $(1,8)$. 
Thus, as we let x-coordinate increase on our line, the
function value increases.
%\pause
\end{itemize}
\end{frame}
%10

\begin{frame}%11
  \frametitle{Explanation continued.}
  \begin{itemize}%[<+-| alert@+>]
\item
\mbl{\bf Conclusion.}
Thus on a line in parametric form, a linear function
increases or decreases with the parameter depending on
the coefficient of the parameter.
  
\item
For the above line, if we consider a different function,
say $g(x,y)=3x-y+2$, then we see that 
$g(1+t,8+3t) = 3(1+t) -(8+3t)+2$ and  this simplifies to 
$g(1+t,8+3t) = -3$.
\item
Thus, a linear function on a line is either constant
at all points or increases in one of the two available
directions and decreases in the opposite direction.

%\pause
\end{itemize}
\end{frame}

%11


\begin{frame}%12
  \frametitle{ Why Corners?}
  \begin{itemize}%[<+-| alert@+>]

\item
Consider the plane region of feasible points that we can plot for our problem.
Where would a linear function become maximum on such a region?
If we take any point \mbl{in the interior} of our region, then we can
draw a little line segment through the point which is still entirely
in the region.
\item
Now if our function is not constant on the line, it would be increasing
in one of the two directions and thus would not be maximum at our given point. 
If by luck, we had chosen a line segment on which the function happens to be
constant, we can choose a different segment through the same point and
make the same argument.
We could not have the same function constant on the second segment as well,
for it is clear that then the function would be identically constant on
the whole plane \mrd{Think why!!} and our problem has a trivial answer:
every point is a maximum point!
%\pause
\end{itemize}
\end{frame}
%12


\begin{frame}%13
  \frametitle{Continued discussion.}
  \begin{itemize}%[<+-| alert@+>]
\item
Thus, our maximum point has to be on the boundary!
\item
It could be on a boundary segment or at a corner.
Note that if it is on a segment, but not at a corner,
then the function has to be constant on the whole boundary segment
(for otherwise we get a contradiction as above).

\item
This is why it is enough to only check the corner points for locating a maximum.
\item
This also explains that if we find two maximum points which are corners,
then the line joining them must form a boundary line,
i.e. they must be adjacent points on the boundary polygon.
\item
If the region is unbounded, then it has boundary lines running
off to infinity and we may find that the maximum point may not
exist in the sense that it has to be a point of infinity on one
such boundary line.
%\pause
\end{itemize}
\end{frame}
%14
\begin{frame}%15
  \frametitle{Some Sample Problems.}
  \begin{itemize}%[<+-| alert@+>]
\item {\bf Problem 3.2.2 by graphing}
The problem is to maximize the profit $P=2x+1.5y$  
subject to $x\ge 0, y\ge 0, 3x+4y\le 1000$  and  $6x+3y\le 1200$
\item We first sketch the lines and find their common point. 
Then we decide on the region.
\item
Note that in the picture below, the inequality $3x+4y\le  1000$
corresponds to the line $BC$ and the inequality $6x+3y \le 1200$
matches the line $AB$.
Their regions both point towards the origin, since the origin
satisfies both of them!
The axes are automatically included with regions pointing
towards the first quadrant.

%\pause
 

\end{itemize}
\end{frame}
%15

\begin{frame}%16
  \frametitle{Problem continued.}
  \begin{itemize}%[<+-| alert@+>]
\item

\begin{tabular}{ll}
\pict{1}{1}{region5_lec7.jpg} & 
\parbox[b]{2.9in}{\vspace{-1.2in}The corners are\\
$O(0,0),A(200,0),B(120,160),C(0,250)$.\\
The values of $2x+1.5y$ at these are $(0,400,480,375)$.}
\end{tabular} 

\item So the maximum is at $B(120,160)$ with maximum value $480$.
%\pause

\end{itemize}
\end{frame}
%16
\begin{frame}%17
  \frametitle{More Solved Problems.}
  \begin{itemize}%[<+-| alert@+>]
\item Consider the problem (similar to B2.8).
Suppose that $x+2y\le 2$ and $y+5x\le 5$ together with $x\ge 0,y\ge 0$.
The maximum value of the function $6x+9y+2$ on the resulting
region occurs $x=\cdots $ and $y=\cdots $.        
The maximum value of the function is $\cdots$. 

\item
We first convert all inequalities to equations
and plot after finding common points.
The equations are:
$$x+2y=2, y+5x=5, x=0,y=0.$$
\item 
\begin{tabular}{ll}
\pict{1}{1}{region6_lec7.jpg} & 
\parbox[b]{2.9in}{\vspace{-1.2in}The corners are\\
$(1,0),(8/9,5/9),(0,1),(0,0)$.\\
The values of $6x+9y+2$ at these are $(8, 37/3, 11, 2)$.}
\end{tabular}
\item So the maximum is at $(8/9,5/9)$ and the maximum value is $37/3$.
%\pause

\end{itemize}
\end{frame}
%17
\begin{frame}%18
  \frametitle{More Solved Problems.}
  \begin{itemize}%[<+-| alert@+>]
\item Consider the problem (similar to B2.9).
Suppose that $y\le 5x,y\ge 3x$ and $x/4+y/5\le 1$ together with $x\ge 0, y\ge 0$.

The maximum value of the function $x+y$ on the resulting
region occurs $x=\cdots $ and $y=\cdots $.        
The maximum value of the function is $\cdots$. 

\item
We first convert all inequalities to equations
and plot after finding common points.
The equations are:
$$y=5x, y=3x, x/4+y/5=1, x=0,y=0.$$
Note that the last two do not contribute to the picture!
\item 
\begin{tabular}{ll}
\pict{1}{1}{region7_lec7.jpg} & 
\parbox[b]{2.9in}{\vspace{-1.2in}The corners are\\
$(0,0),(4/5,4),(20/17,60/17)$.\\
The values of $x+y$ at these are $(0,24/5=4.8,80/17=4.7059)$.}
\end{tabular}
\item So the maximum is at $(4/5,4)$ and the maximum value is $4.8$.
%\pause

\end{itemize}
\end{frame}


\begin{frame} %2

  \frametitle{Old problem with a new method.}
 \begin{itemize}%[<+-| alert@+>]  
\item  We recall the problem we solved using the graphic method.
 $$\mbox{ Maximize } P=2x+1.5y \mbox{ s.t. } 3x+4y\le 1000,
 6x+3y\le 1200, x,y\ge 0.$$

\item
 This was the sketch of the feasible region.
\centerline{\pict{1.5}{1.5}{prob1_lec8.jpg}}
 
\item
The corner points were $O(0,0), A(200,0), B(120,160)$ and $C(0,250)$.
By checking the function values at the four corner points we found
the maximum value at B.

\end{itemize}
\end{frame}

%2
%

\begin{frame}%3
  \frametitle{Example continued.}
  \begin{itemize}%[<+-| alert@+>]

\item We begin by setting up a problem table which will become useful
later.
$$
\begin{array}{rr|r}
x & y & RHS \\
3 & 4 & 1000\\
6 & 3 & 1200 \\
2 & 1.5 & \\
\end{array}
$$

\item Note that $x,y\ge 0$ is not listed, since it is always assumed.
We are listing inequalities which are always assumed to be of the form
$\le $ with the last row giving the function coefficients.

\item We next convert these to equations and write a proper augmented
matrix called a simplex tableaux.

\end{itemize}
\end{frame}

%3
\begin{frame}%4
  \frametitle{Simplex tableaux.}
  \begin{itemize}%[<+-| alert@+>]
\item We introduce new variables called slack variables so that an
inequality $3x+4y\le 1000$ is replaced by $3x+4y+u=1000$ with the
understanding that $u\ge 0$.

Similarly, the second inequality becomes $6x+3y+v=1200$ with $v\ge 0$.


\item The maximized function declaration  is rewritten as
$-2x-1.5y+P=0$.
\mbl{\bf Note the changes of signs.}

\item This sets up an initial tableaux:
$$
\begin{array}{rrrrr|r}
x & y & u & v & P & RHS\\\hline
3 & 4 & 1 & 0 & 0 & 1000\\
6 & 3 & 0 & 1 & 0 & 1200 \\\hline
-2 & -1.5 & 0 & 0 & 1 & 0\\
\end{array}
$$



\end{itemize}

\end{frame}

%4


\begin{frame}%5
  \frametitle{Basic Solution.}
  \begin{itemize}%[<+-| alert@+>]
 
\item 
The simplex tableaux always has a set of unit columns (columns of the
Identity matrix) which exactly fill up an identity matrix after possible
rearrangement.

\item
\mbl{The corresponding variables on top of these columns are said to be basic
variables and form a
basis} for the current tableaux.

The number of basic variables must be the same as the total number of rows.

\item Corresponding to the basis, we have a \mrd{basic solution} to the
current tableaux. It is obtained by setting the \mbl{ non basic
variables } to zero value and reading off the solutions of the basic
variables from all the equations.

\item Thus, for our first tableaux above, the basis is $u,v,P$ and the
basic solution is:
$$(x,y,u,v,P)=(0,0,1000,1200,0).$$

\end{itemize}
\end{frame}

%5

\begin{frame}%6
  \frametitle{Modification of the Simplex tableaux.}
  \begin{itemize}%[<+-| alert@+>]
 
\item Note that our original variables $(x,y)$ have values $(0,0)$ and
thus, this corresponds to the corner point $O$ of our original feasible
region.

\item If we solve the last equation for its basic variable $P$, then we
have:
$$P=2x+1.5y$$and this says that increasing the values of $x$ or $y$ will
increase the value of $P$.

\mbl{Note that this is signified by the negative numbers $-2,-1.5$ in the
tableaux.}
\item So we try to make one of them, say $x$ as a basic variable. This
means the $x$ column has to become a unit column.

Where shall it have its $1$?

\end{itemize}
\end{frame}

%6

\begin{frame}%7
  \frametitle{Choosing a Pivot.}
  \begin{itemize}%[<+-| alert@+>]
 

\item
You may try and make the pivot at any available non zero entry in the
$x$ column. However, some choices will produce negative entries in the
last column. This would mean that the new basic solution will have
negative values for some basic variables. This is not allowed, since by
our set up all the variables have non negative values.

\item There is a simple test which will guarantee that we won't get in
trouble.

Check the ratios obtained by dividing \mrd{ non zero entries } in the
$x$-column into the corresponding entries in RHS and choosing the one
which gives a minimum value.

\item In our example, the choices are $\frac{1000}{3}$ and
$\frac{1200}{6}=200$ respectively. Note that we do not use negative
entry $-2$. The smallest ratio is $200$ in the second row.


\end{itemize}
\end{frame}

%7

\begin{frame}%8
 \frametitle{The Cleanup Row Operations.}
  \begin{itemize}%[<+-| alert@+>]

\item Thus, we use the $(2,1)$ entry $6$ to clean out all entries above
and below.

\item The row oeprations $R_3-\frac{-2}{6}R_2$, $R_1-\frac{3}{6}R_2$ and
finally $\frac{1}{6}R_2$ produce the following new tableaux.
\item
$$
\begin {array}{rrrrr|r}
x & y & u & v & P & RHS\\\hline
0&  5/2  & \mbl{1}&-1/2&0&400\\
\mbl{1} &1/2&0&1/6&0&200\\\hline
0&-1/2&0&1/3&\mbl{1}&400
\end {array}
$$


\item Note that the new basis is now $x,u,P$ and hence the new basic
solution is:
$$(x,y,u,v,P) = (200,0,400,0,400).$$

\end{itemize}
 
\end{frame}

%8

\begin{frame}%9
 \frametitle{Continued Pivot Operations.}
  \begin{itemize}%[<+-| alert@+>]
 
\item We note that in terms of $x,y$ variables we are at a point
$(200,0)$ which was our point $A$. Thus we have marched from $O$ to $A$
on our feasible graph. We also have improved our function value from $0$
to $400$.

\item We notice the negative number $-\frac{1}{2}$ in the function row
and this tells us that it would help to put $y$ in the set of basic
variables. Explicitly, the last equation is
$P=\frac{y}{2}-\frac{v}{3}+400$, so $y$ wants to get bigger!

\item We check the pivot ratios for the second ($y$) column:
$$\frac{400}{5/2} = 160 \mbox{ and } \frac{200}{1/2}=400.$$
Thus the new pivot must be at $(1,1)$ position.

\end{itemize}
 
\end{frame}

%9

\begin{frame}%10
  \frametitle{Cleanup and end.}
  \begin{itemize}%[<+-| alert@+>]
\item
The pivot operations are $R_2-\frac{1}{5}R_1$ and $R_3+\frac{1}{5}R_1$
and produce:
$$
\begin {array}{rrrrr|r}
x & y & u & v & P & RHS\\\hline
0&1&2/5&-1/5&0&160\\
1&0&-1/5&{\frac {4}{15}}&0&120\\\hline
0&0&1/5&{\frac {7}{30}}&1&480
\end {array} 
$$

\item The new basic is clearly $x,y,P$ and the new basic solution is:
$$(x,y,u,v,P)=(120,160,0,0,480).$$

Note that this matches our point $B$ in the graph of feasible points.


\end{itemize}
\end{frame}
%10


\begin{frame}%11
  \frametitle{Conclusion.}
  \begin{itemize}%[<+-| alert@+>]
\item Note that the last equation is now
$P=-\frac{u}{5}-\frac{7v}{30}+480$ so the non basic variables $u,v$ will not
increase its value if they enter the basis.

This is visible since the last row has only non negative coefficients
under the variable columns.

\mbl{So, we are done!!!}

\item Thus we have a clear strategy called the Simplex algorithm which
we outline next.

\end{itemize}
\end{frame}

%11

\begin{frame}%12
  \frametitle{ The Simplex Algorithm.}
  \begin{itemize}%[<+-| alert@+>]

\item We only consider a simplified set up outlined in the book. Further
complications are reserved for higher level courses.
\item Thus, we assume that the starting tableaux has a basis of slack
variables and the function $P$ with the RHS entries all non negative.
This, in turn means that our maximization problem has all inequalities
of the form $\le$, except for the assumption that all variables are non
negative.
\item If there are no negative entries in the last equation (function
row), then we are done and the current basic solution is the optimal
solution. Our basis always has the function variable $P$.
\item If there is some negative entry in the function row then we call
its column the pivot column.

We turn it into a unit column as described below.

\end{itemize}
\end{frame}
%12

\begin{frame}%13
  \frametitle{Algorithm Continued.}
  \begin{itemize}%[<+-| alert@+>]
\item Compute the pivot ratios, i.e. the ratios obtained by dividing
\mbl{positive} entries in the pivot column nto the corresponding RHS
entries.

\item Pick the smallest among the ratios and pick the corresponding
entry in the pivot column as the chosen pivot.

\item Do the standard row opertaions to make the chosen pivot $1$ and
all other entries in the pivot column zero.
\item This has the effect of entering the pivot column variable into the
basis. The old basic variable which had its pivot in the current pivot
row gets kicked out of the basis.

\item Inspect the new tableux for negative entries in the function row
and continue.

\end{itemize}
\end{frame}
%14
\begin{frame}%15
  \frametitle{What can go wrong?}
  \begin{itemize}%[<+-| alert@+>]
\item \mbl{The unbounded case.}
Several problems can come up during the above process.
The first is that in the pivot column, we may not have any positive
entry to pick as a new pivot. In this case, it can be shown that our
problem is unsolvable because the feasible region is unbounded and our
function values can grow indefinitely.

\item It may happen that the chosen pivot has a pivot ratio $0$. In this
case, the basis exchange does not increase the function value. This may
continue for several steps so that we may get into a loop of basis
choices.

This does not happen for a small number of variables and is always
accompanied by existence of multiple pivot choices. It can be shown that
if we make random choices among multiple pivot choices, then the looping
can be escaped.



\end{itemize}
\end{frame}

\begin{frame} %2

  \frametitle{Standard Optimization Problems.}
 \begin{itemize}%[<+-| alert@+>]  
\item  A \mbl{standard maximization problem} can be conveniently
described in matrix form as follows.

Maximize $P=CX$ subject to $AX\le B$ and $X\ge 0$.

Here, $X$ is a column of the variables used in the problem, $C$ is a row
vector, so that $P=CX$ is a linear function of $X$. It is often
similar to a profit function, hence the letter $P$.

\item 
Moreover, in the current course we assume that $B\ge 0$. This insures
that the choice $X=0$ satisfies all the inequalities, i.e. is a feasible
solution.

Problems can be analyzed without this assumption, but we won't try to
solve them not in this course.

\item
Here is an example (4.1 Example 3): Take
$$A=\bmatr{3}{2 & 1 & 2\\ 2 & 4 & 1 \\ 1 & 2 & 3},~
B=\bmatr{1}{14\\26\\28} \mbox{ and } C = \bmatr{3}{2 & 2 & 1}.$$


\end{itemize}
%\pause 
\end{frame}

%2
%

\begin{frame}%3
  \frametitle{Example continued.}
  \begin{itemize}%[<+-| alert@+>]

\item We construct a problem table (or tableau) recording all the coefficients
as follows:
$$
\left[
\begin{array}{lll|l}
x & y & z & Constants \\\hline
2 & 1 & 2 & 14\\
2 & 4 & 1 & 26\\
1 & 2 & 3 & 28\\\hline
2 & 2 & 1 & *\\
\end{array}
\right]
\mbox{ or symbolically }
\left[
\begin{array}{l|l}
X & Constants \\\hline
A & B\\\hline
C & *\\
\end{array}
\right].
$$
\item We are using the title ``constants'' to match the book notation.
When we write the Simplex tableau, it may be changed to RHS, since in
the tableau, we have equations, not just inequalities.


\end{itemize}
%\pause
\end{frame}
%3


\begin{frame}%4
  \frametitle{Minimization tableau.}
  \begin{itemize}%[<+-| alert@+>]

\item There is a natural \mbl{dual problem } associated with this table
and it has a dual problem table (tableau) as follows:
$$
\left[
\begin{array}{lll|l}
u & v & w & Constants \\\hline
2 & 2 & 1  & 2\\
1 & 4 & 2 & 2\\
2 & 1 & 3 & 1\\\hline
14 & 26 & 28& *\\
\end{array}
\right]
\mbox{ or symbolically }
\left[
\begin{array}{l|l}
Y & Constants \\\hline
A' & C'\\\hline
B' & *\\
\end{array}
\right].
$$

The matrices $A',B',C'$ are the transposes of $A,B,C$ respectively. Thus
the whole table is simply a transpose.
\item Our dual problem is described as follows:
Minimize $14u+26v+28w$ subject to the conditions
$$2u+2v+w\ge 2, u+4v+2w\ge 2, 2u+v+3w\ge 1 \mbox{ and } u,v,w\ge 0.$$



\end{itemize}
%\pause
\end{frame}

%4


\begin{frame}%5
  \frametitle{Duality theorem.}
  \begin{itemize}%[<+-| alert@+>]

\item The dual problem is described as a minimization problem as
follows. We let $Y$ be a row vector of variables, equal in number to the
number of rows of $A$.
$$\mbox{ Minimize } YB \mbox{ subject to } YA\ge C \mbox{ and } Y\ge
0.$$

Compare this with the inequalities above by taking $Y=\bmatr{3}{u & v &
w}$.

\item The reason to write $Y$ as a row rather than a column is somewhat technical
and would
be clarified below.

\item The amazing theorem called the ``duality theorem'' states that any
solution of the original maximization problem by the Simplex Algorithm
produces a solution to its dual minimization problem by simply reading
the final tableau.

We describe this next.


\end{itemize}
%\pause
\end{frame}

%5



\begin{frame}%6
  \frametitle{The solution from the algorithm.}
  \begin{itemize}%[<+-| alert@+>]
 
\item
For our maximization problem above, we record the starting and the final
tableaux and then show how to interpret them.

\item Starting tableau:
$$\left[
\begin{array}{rrrrrrr|r}
x & y & z & u & v & w & P & RHS\\\hline
2 & 1 & 2 & 1 & 0 & 0 & 0 & 14\\
2 & 4 & 1 & 0 & 1 & 0 & 0 & 26\\
1 & 2 & 3 & 0 & 0 & 1 & 0 & 28\\\hline
-2 & -2 & -1 & 0 & 0 & 0 & 1 & 0 \\
\end{array}
\right].
$$

\item End tableau:
$$\left[
\begin{array}{rrrrrrr|r}
x & y & z & u & v & w & P & RHS\\\hline
1 & 0 & 7/6 & 2/3 & -1/6 & 0 & 0 & 5\\
0 & 1 & -1/3  & -1/3 & 1/3 & 0 & 0 & 4\\
0 & 0 & 5/2 & 0 & -1/2 & 1 & 0 & 15\\\hline
0 & 0 & 2/3 & 2/3 & 1/3 & 0 & 1 & 18 \\
\end{array}
\right].
$$

\end{itemize}
%\pause
\end{frame}

%6

\begin{frame}%7
  \frametitle{The interpretation of the solution.}
  \begin{itemize}%[<+-| alert@+>]
 

\item
Inspection of the final tableau says that the final basis is $x,y,w,P$
and hence the final basic solution is:
$$(x,y,z,u,v,w,P)=(5,4,0,0,0,15,18).$$

\item The Voodoo Principle also tells us the actual row transformations
that we performed. This information is read  from the $4\times 4$
matrix under the variables $u,v,w,P$.

\item We know that the row transformations can be performed by
multiplying the original $4\times 8$ matrix by some $4\times 4$ matrix
on the left.
We see that this transformation matrix  must be

$$
\left[
\begin{array}{l|l}
M & 0 \\\hline
Y & 1\\
\end{array}
\right]
~=~
\left[
\begin{array}{rrr|r}
2/3 & -1/6 & 0 & 0\\
-1/3 & 1/3 & 0 & 0\\
0 & -1/2 & 1 & 0\\\hline
2/3 & 1/3 & 0 & 1\\
\end{array}
\right]
.$$

\end{itemize}
%\pause
\end{frame}



%7

\begin{frame}%8
 \frametitle{Interpretation continued.}
  \begin{itemize}%[<+-| alert@+>]

\item The first part of the last row $Y=\bmatr{3}{2/3 & 1/3& 0}$ tells
us that we must have multiplied the first three rows by $2/3, 1/3,0$
respectively and added to the last row.

\item By looking at the the entries at the foot of the $x,y,z$ columns,
we deduce that
$$\bmatr{3}{0 & 0 & 2/3} = -C+YA$$
since the original entries were $-C = \bmatr{3}{-2 & -2 & -1}$ and we
added $YA$ to it.
Thus $YA\geq C$.

\item By looking at the last entry in the bottom row, we know that it
was $0$ and we have added $YB$ to it.

\item Thus, we have
$$Y\ge 0, YB = (2/3)\cdot(14)+ (1/3)\cdot(26)+(0)\cdot(28) = 18.$$

\item Thus $Y$ is a feasible solution to the dual problem.

\end{itemize}
%\pause
\end{frame}

%8


\begin{frame}%9
 \frametitle{Why do we have the dual problem solved?}
  \begin{itemize}%[<+-| alert@+>]
 
\item Recall that the two Linear Programming Problems (LPP)  are:
$$\mbox{ Maximize: } P=CX \mbox{ s.t. } X\ge 0, AX\le B $$
and
$$\mbox{ Minimize: } Q=YB \mbox{ s.t. } Y\ge 0, YA\ge C .$$
\mbl{Here we name the second function $Q$ instead of $C$ since $C$ is used
for the coefficients of $P$.}

\item Recall that by a feasible solution to either problem we mean a
solution which satisfies all the inequalities, but may not give the
maximum or minimum.

\item If $X_0,Y_0$ are feasible solutions to the two problems
respectively, then we see that
$Y_0B \geq Y_0AX_0\ge CX_0$. Thus the function value $Q_0=Y_0B$ is always
bigger than or equal to the function value $P_0=CX_0$.



\end{itemize}
%\pause
\end{frame}

%9



\begin{frame}%10
  \frametitle{Proof of Duality.}
  \begin{itemize}%[<+-| alert@+>]

\item Thus, if $X_0$ and $Y_0$ are feasible solutions to the
maximization and its dual minimization problems respectively and if
$$P_0=CX_0 = Y_0B=Q_0$$
then \mbl{ both must be simultaneously the optimum values for the
respective problems}, hence the solutions of both the problems at once!

\item Thus for our dual problems
$X_0=(5,4,0)$ and $Y_0=(2/3,1/3,0)$ are the respective solutions of the
maximization and the minimization problems with a common function value
$18$.

\item \mrd{Warning!} Note that the $Y$ values are read at the foot of
the original slack variables, but they are not the values of the basic
solution for the  slack variables  corresponding to the maximization
problem. 


\end{itemize}
%\pause
\end{frame}
%10


\begin{frame}%11
  \frametitle{How to handle optimization problems?}
  \begin{itemize}%[<+-| alert@+>]
\item
Recall that we always assume all our variables to be non negative in
this course. In the following discussion, we are only discussing the
remaining inequalities; we shall call them essential inequalities.

\item 
If our essential inequalities are of $\le$ type with non negative RHS,
then we write it as a maximization problem and \mbl{solve with the Simplex
algorithm.} (If we happen to be minimizing a function, we can always
maximize its negative instead!)

\item
If our essential inequalities are of $\ge$ type with non negative RHS,
then we write it as a minimization problem and \mbl{solve its dual with
the Simplex algorithm.} Then read off the solution under the slack columns as
shown above. (If we happen to be maximizing a function, we can always
minimize its negative instead!)

\item \mbl{Terminology.} The problem we wish to solve is always called
the  primal problem and its dual is the dual problem.

\end{itemize}
%\pause
\end{frame}

%11

\begin{frame}%12
  \frametitle{ When do we fail?}
  \begin{itemize}%[<+-| alert@+>]

\item Recall that if we have a negative entry in the last row of a
simplex tableau but no suitable pivot above because all such entries are
less than or equal to zero, then our maximization problem is unbounded
and has no solution.

For the dual minimization problem, we can claim that there is no
feasible solution. This means that its feasible region is empty!

\item We can also have a case where the dual minimization problem is
unbounded, but then the primal maximization problem shall have no
feasible solution! This cannot occur  under our standardness assumption.

You will meet this in higher courses.

\end{itemize}
%\pause
\end{frame}
%12
