%%%%%%%%%%%%%%%%%%%%%%%%%%%%%%%%%%%%%%%%%%%%%%%%%%%%%%%%%%%%%%%%%%%%%%%%%%%%%%%%%%
\begin{frame}[fragile]\frametitle{}
\begin{center}
{\Large Numbers}
\end{center}
\end{frame}

%%%%%%%%%%%%%%%%%%%%%%%%%%%%%%%%%%%%%%%%%%%%%%%%%%%%%%%%%%%
 \begin{frame}[fragile]\frametitle{Numbers}
\begin{itemize}
\item $ x = 3 \Rightarrow \mathbb{N}$: Natural numbers: 1,2,3,\ldots
\item $ x + 5 = 3 \Rightarrow \mathbb{Z}$: Integers: -1,0,1,2,3,\ldots
\item $ 2x = 3 \Rightarrow \mathbb{Q}$: Rational numbers, ratios: $\frac{3}{2},\frac{1}{3}$,\ldots
\item $ x^2 = 2 \Rightarrow \mathbb{P}$: Irrational numbers, cannot be expressed as ratios: $\sqrt{2},\sqrt[3]{2}$,\ldots.
\item $\pi, e \Rightarrow \mathbb{R}$: Real numbers, these cannot be represented by polynomials, cannot be roots, etc.
\item $ x^2 +1 = 0 \Rightarrow \mathbb{C}$: Complex numbers: $i,2+3i$,\ldots.All polynomials have roots in $\mathbb{C}$
\end{itemize}

$\mathbb{N} \subseteq \mathbb{Z} \subseteq \mathbb{Q} \subseteq \mathbb{P} \subseteq \mathbb{R} \subseteq \mathbb{C} $

Note: all the coefficients in the equation are $\mathbb{N}$ but the resultant $x$ is of different types.
\end{frame}