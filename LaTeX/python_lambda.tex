%%%%%%%%%%%%%%%%%%%%%%%%%%%%%%%%%%%%%%%%%%%%%%%%%%%%%%%%%%%%%%%%%%%%%%%%%%%%%%%%%%
\begin{frame}[fragile]\frametitle{}
\begin{center}
{\Large Lambda Functions}
\end{center}
\end{frame}

%%%%%%%%%%%%%%%%%%%%%%%%%%%%%%%%%%%%%%%%%%%%%%%%%%%%%%%%%%%%%%%%%%%%%%%%%%%%%%%%%%%
\begin{frame}[fragile]\frametitle{Lambda Intro}

    \begin{itemize}
    \item  lambda operator or lambda function is used for creating small, one-time and anonymous function objects in Python.
    \item lambda operator can have any number of arguments, but it can have only one expression. 
    \item It cannot contain any statements and it returns a function object which can be assigned to any variable.
    \end{itemize}
    \begin{lstlisting}
lambda arguments : expression
\end{lstlisting}
\end{frame}

%%%%%%%%%%%%%%%%%%%%%%%%%%%%%%%%%%%%%%%%%%%%%%%%%%%%%%%%%%%%%%%%%%%%%%%%%%%%%%%%%%%
\begin{frame}[fragile]\frametitle{Lambda Intro}
Eg. add expects two arguments x and y and returns their sum.
\begin{lstlisting}
 def add(x, y): 
    return x + y

add(2, 3) # Output: 5
\end{lstlisting}
lambda function:
\begin{lstlisting}
add = lambda x, y : x + y 
  
print add(2, 3) # Output: 5
\end{lstlisting}
\end{frame}


%%%%%%%%%%%%%%%%%%%%%%%%%%%%%%%%%%%%%%%%%%%%%%%%%%%%%%%%%%%%%%%%%%%%%%%%%%%%%%%%%%%
\begin{frame}[fragile]\frametitle{Lambda Intro}

    \begin{itemize}
    \item  In \lstinline|lambda x, y: x + y;| x and y are arguments to the function and x + y is the expression which gets executed and its values is returned as output.
\item \lstinline|lambda x, y: x + y| returns a function object which can be assigned to any variable, in this case function object is assigned to the add variable.

    \end{itemize}
    \begin{lstlisting}
type (add) # Output: function
\end{lstlisting}
\end{frame}


%%%%%%%%%%%%%%%%%%%%%%%%%%%%%%%%%%%%%%%%%%%%%%%%%%%%%%%%%%%%%%%%%%%%%%%%%%%%%%%%%%%
\begin{frame}[fragile]\frametitle{Lambda is a tool for building functions}

    \begin{itemize}
    \item That means that Python has two tools for building functions: def and lambda.
    \item Here's an example. You can build a function in the normal way, using def, like this:
        \begin{lstlisting}
def square_root(x): return math.sqrt(x)
\end{lstlisting}
\item or you can use lambda:
        \begin{lstlisting}
square_root = lambda x: math.sqrt(x)
\end{lstlisting}
\item Here are a few other interesting examples of lambda:
        \begin{lstlisting}
sum = lambda x, y:   x + y   #  def sum(x,y): return x + y
out = lambda   *x:   sys.stdout.write(" ".join(map(str,x)))
lambda event, name=button8.getLabel(): self.onButton(event, name)
\end{lstlisting}
    \end{itemize}

\end{frame}


%%%%%%%%%%%%%%%%%%%%%%%%%%%%%%%%%%%%%%%%%%%%%%%%%%%%%%%%%%%%%%%%%%%%%%%%%%%%%%%%%%%
\begin{frame}[fragile]\frametitle{Why do we need lambda?}

    \begin{itemize}
    \item  Actually, we don't absolutely need lambda; we could get along without it. 
    \item But there are certain situations where it makes writing code a bit easier, and the written code a bit cleaner. 
    \item What kind of situations?: Situations in which (a) the function is fairly simple, and (b) it is going to be used only once.
    \end{itemize}

\end{frame}

%%%%%%%%%%%%%%%%%%%%%%%%%%%%%%%%%%%%%%%%%%%%%%%%%%%%%%%%%%%%%%%%%%%%%%%%%%%%%%%%%%%
\begin{frame}[fragile]\frametitle{Why do we need lambda?}

Is this better?
        \begin{lstlisting}
def __init__(self, parent):
    """Constructor"""
    frame = tk.Frame(parent)
    frame.pack()
 
    btn22 = tk.Button(frame,
        text="22", command=lambda: self.printNum(22))
    btn22.pack(side=tk.LEFT)
 
    btn44 = tk.Button(frame,
        text="44", command=lambda: self.printNum(44))
    btn44.pack(side=tk.LEFT)
\end{lstlisting}

\end{frame}


%%%%%%%%%%%%%%%%%%%%%%%%%%%%%%%%%%%%%%%%%%%%%%%%%%%%%%%%%%%%%%%%%%%%%%%%%%%%%%%%%%%
\begin{frame}[fragile]\frametitle{Why do we need lambda?}

Or is this better?
        \begin{lstlisting}
sum = lambda x, y:   x + y   #  def sum(x,y): return x + y
out = lambda   *x:   sys.stdout.write(" ".join(map(str,x)))
lambda event, name=button8.getLabel(): self.onButton(event, name)
\end{lstlisting}

\end{frame}

%%%%%%%%%%%%%%%%%%%%%%%%%%%%%%%%%%%%%%%%%%%%%%%%%%%%%%%%%%%%%%%%%%%%%%%%%%%%%%%%%%%
\begin{frame}[fragile]\frametitle{Why is lambda so confusing?}
Becuase of some constraints:
    \begin{itemize}
    \item  Assignment statements cannot be used in lambda. In Python, assignment statements don't return anything, not even None (null).
    \item Simple things such as mathematical operations, string operations, list comprehensions, etc. are OK in a lambda.
    \item Function calls are expressions. It is OK to put a function call in a lambda, and to pass arguments to that function. 
    \item In Python 3, print became a function, so in Python 3+, print() can be used in a lambda.
\item Even functions that return None, like the print function in Python 3, can be used in a lambda.
    \item   lambda specification does not contain a return statement.
    \item Why only one expression? Why not multiple expressions? 
    \end{itemize}

\end{frame}


%%%%%%%%%%%%%%%%%%%%%%%%%%%%%%%%%%%%%%%%%%%%%%%%%%%%%%%%%%%%%%%%%%%%%%%%%%%%%%%%%%%
\begin{frame}[fragile]\frametitle{Lambda Usage}

    \begin{itemize}
    \item  Mostly lambda functions are passed as parameters to a function which expects a function objects as parameter like map, reduce, filter functions
	\item map
	    \begin{lstlisting}
map(function_object, iterable1, iterable2,...)
\end{lstlisting}
\item map functions expects a function object and any number of iterables like list, dictionary, etc.
\item It executes the function\_object for each element in the sequence and returns a list of the elements modified by the function object.
    \end{itemize}

\end{frame}

%%%%%%%%%%%%%%%%%%%%%%%%%%%%%%%%%%%%%%%%%%%%%%%%%%%%%%%%%%%%%%%%%%%%%%%%%%%%%%%%%%%
\begin{frame}[fragile]\frametitle{Lambda Example}
Without lambda:
	    \begin{lstlisting}
def multiply2(x):
  return x * 2
    
map(multiply2, [1, 2, 3, 4]) # Output [2, 4, 6, 8]
\end{lstlisting}
With lambda:
	    \begin{lstlisting}
map(lambda x : x*2, [1, 2, 3, 4]) #Output [2, 4, 6, 8]
\end{lstlisting}
\end{frame}

%%%%%%%%%%%%%%%%%%%%%%%%%%%%%%%%%%%%%%%%%%%%%%%%%%%%%%%%%%%%%%%%%%%%%%%%%%%%%%%%%%%
\begin{frame}[fragile]\frametitle{Lambda Example}
Iterating over a dictionary using map and lambda
	    \begin{lstlisting}
dict_a = [{'name': 'python', 'points': 10}, {'name': 'java', 'points': 8}]
map(lambda x : x['name'], dict_a) # Output: ['python', 'java']
map(lambda x : x['points']*10,  dict_a) # Output: [100, 80]
map(lambda x : x['name'] == "python", dict_a) # Output: [True, False]
\end{lstlisting}
In the above example, each dict of dict\_a will be passed as parameter to the lambda function. Result of lambda function expression for each dict will be given as output.
\end{frame}

%%%%%%%%%%%%%%%%%%%%%%%%%%%%%%%%%%%%%%%%%%%%%%%%%%%%%%%%%%%%%%%%%%%%%%%%%%%%%%%%%%%
\begin{frame}[fragile]\frametitle{Multiple iterables to the map function}
We can pass multiple sequences to the map functions as shown below:
	    \begin{lstlisting}
list_a = [1, 2, 3]
list_b = [10, 20, 30]
  
map(lambda x, y: x + y, list_a, list_b) # Output: [11, 22, 33]
\end{lstlisting}
Here, each ith element of list\_a and list\_b will be passed as argument to the lambda function.
\end{frame}

%%%%%%%%%%%%%%%%%%%%%%%%%%%%%%%%%%%%%%%%%%%%%%%%%%%%%%%%%%%%%%%%%%%%%%%%%%%%%%%%%%%
\begin{frame}[fragile]\frametitle{map}
    \begin{itemize}
    \item  In Python3, map function returns an iterator or map object which gets lazily evaluated. 
    \item Just like zip function is lazily evaluated.
    \item Neither we can access the elements of the map object with index nor we can use len() to find the length of the map object
\item We can force convert the map output i.e. the map object to list as shown below:
    \end{itemize}
    	    \begin{lstlisting}
map_output = map(lambda x: x*2, [1, 2, 3, 4])
print(map_output) # Output: map object: <map object at 0x04D6BAB0>
list_map_output = list(map_output)
print(list_map_output) # Output: [2, 4, 6, 8]
\end{lstlisting}
\end{frame}

%%%%%%%%%%%%%%%%%%%%%%%%%%%%%%%%%%%%%%%%%%%%%%%%%%%%%%%%%%%%%%%%%%%%%%%%%%%%%%%%%%%
\begin{frame}[fragile]\frametitle{Exercise}
Write a program which can map() to make a list whose elements are square of numbers between 1 and 20 (both included).



\end{frame}

%%%%%%%%%%%%%%%%%%%%%%%%%%%%%%%%%%%%%%%%%%%%%%%%%%%%%%%%%%%%%%%%%%%%%%%%%%%%%%%%%%%
\begin{frame}[fragile]\frametitle{Solution}
\begin{lstlisting}
squaredNumbers = map(lambda x: x**2, range(1,21))
print(squaredNumbers)
\end{lstlisting}
\end{frame}


%%%%%%%%%%%%%%%%%%%%%%%%%%%%%%%%%%%%%%%%%%%%%%%%%%%%%%%%%%%%%%%%%%%%%%%%%%%%%%%%%%%
\begin{frame}[fragile]\frametitle{reduce}
    \begin{itemize}
    \item  The function reduce(func, seq) continually applies the function func() to the sequence seq.
    \item It returns a single value. 
    \item At first the first two elements of seq will be applied to func
    \item In the next step func will be applied on the previous result and the third element of the list
    \item Continue like this until just one element is left and return this element as the result of reduce()
    \end{itemize}
    	    \begin{lstlisting}
>>> reduce(lambda x,y: x+y, [47,11,42,13])
113
\end{lstlisting}
\end{frame}


%%%%%%%%%%%%%%%%%%%%%%%%%%%%%%%%%%%%%%%%%%%%%%%%%%%%%%%%%%%%%%%%%%%%%%%%%%%%%%%%%%%
\begin{frame}[fragile]\frametitle{filter}
    \begin{itemize}
    \item  filter function expects two arguments, function\_object and an iterable. function\_object returns a boolean value. 
    \item function\_object is called for each element of the iterable and filter returns only those element for which the function\_object returns true.
    \item Like map function, filter function also returns a list of element. Unlike map function filter function can only have one iterable as input.
    \end{itemize}
    Even number using filter function:
    	    \begin{lstlisting}
a = [1, 2, 3, 4, 5, 6]
filter(lambda x : x % 2 == 0, a) # Output: [2, 4, 6]
\end{lstlisting}
Filter list of dicts
    	    \begin{lstlisting}
dict_a = [{'name': 'python', 'points': 10}, {'name': 'java', 'points': 8}]
filter(lambda x : x['name'] == 'python', dict_a) # Output: [{'name': 'python', 'points': 10}]
\end{lstlisting}
\end{frame}


%%%%%%%%%%%%%%%%%%%%%%%%%%%%%%%%%%%%%%%%%%%%%%%%%%%%%%%%%%%%%%%%%%%%%%%%%%%%%%%%%%%
\begin{frame}[fragile]\frametitle{filter}
    \begin{itemize}
    \item  Similar to map, filter function in Python3 returns a filter object or the iterator which gets lazily evaluated. 
    \item Neither we can access the elements of the filter object with index nor we can use len() to find the length of the filter object.
    \end{itemize}
\begin{lstlisting}
list_a = [1, 2, 3, 4, 5]
filter_obj = filter(lambda x: x % 2 == 0, list_a) # filter object <filter at 0x4e45890>
even_num = list(filter_obj) # Converts the filer obj to a list
print(even_num) # Output: [2, 4]
\end{lstlisting}
\end{frame}

%%%%%%%%%%%%%%%%%%%%%%%%%%%%%%%%%%%%%%%%%%%%%%%%%%%%%%%%%%%%%%%%%%%%%%%%%%%%%%%%%%%
\begin{frame}[fragile]\frametitle{Exercise}
Write a program which can filter() to make a list whose elements are even number between 1 and 20 (both included).


\end{frame}

%%%%%%%%%%%%%%%%%%%%%%%%%%%%%%%%%%%%%%%%%%%%%%%%%%%%%%%%%%%%%%%%%%%%%%%%%%%%%%%%%%%
\begin{frame}[fragile]\frametitle{Solution}
\begin{lstlisting}
Numbers = filter(lambda x: x%2==0, range(1,21))
print(Numbers)
\end{lstlisting}
\end{frame}


%%%%%%%%%%%%%%%%%%%%%%%%%%%%%%%%%%%%%%%%%%%%%%%%%%%%%%%%%%%%%%%%%%%%%%%%%%%%%%%%%%%
\begin{frame}[fragile]\frametitle{sort}
The sorted() have a key parameter to specify a function to be called on each list element prior to making comparisons.
\begin{lstlisting}
>>> death = [
    ('James', 'Dean', 24),
    ('Jimi', 'Hendrix', 27),
    ('George', 'Gershwin', 38),
]
>>> sorted(death, key=lambda age: age[2])
[('James', 'Dean', 24), ('Jimi', 'Hendrix', 27), ('George', 'Gershwin', 38)]
\end{lstlisting}
\end{frame}


%%%%%%%%%%%%%%%%%%%%%%%%%%%%%%%%%%%%%%%%%%%%%%%%%%%%%%%%%%%%%%%%%%%%%%%%%%%%%%%%%%%
\begin{frame}[fragile]\frametitle{Exercise}
Write a program which can map() and filter() to make a list whose elements are square of even number in [1,2,3,4,5,6,7,8,9,10].

Hints:

Use map() to generate a list.
Use filter() to filter elements of a list.
Use lambda to define anonymous functions.

\end{frame}

%%%%%%%%%%%%%%%%%%%%%%%%%%%%%%%%%%%%%%%%%%%%%%%%%%%%%%%%%%%%%%%%%%%%%%%%%%%%%%%%%%%
\begin{frame}[fragile]\frametitle{Solution}
\begin{lstlisting}
li = [1,2,3,4,5,6,7,8,9,10]
evenNumbers = map(lambda x: x**2, filter(lambda x: x%2==0, li))
print(evenNumbers)
\end{lstlisting}
\end{frame}




%%%%%%%%%%%%%%%%%%%%%%%%%%%%%%%%%%%%%%%%%%%%%%%%%%%%%%%%%%%%%%%%%%%%%%%%%%%%%%%%%%%
\begin{frame}[fragile]\frametitle{Assignments}
    	    \begin{lstlisting}
L1 = [1, 7, 4, -2, 3]

print sorted(L1, key=lambda x: abs(x))
print sorted(L1, key = lambda x: -x)
\end{lstlisting}
\end{frame}

