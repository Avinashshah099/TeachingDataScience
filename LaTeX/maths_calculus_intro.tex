%%%%%%%%%%%%%%%%%%%%%%%%%%%%%%%%%%%%%%%%%%%%%%%%%%%%%%%%%%%%%%%%%%%%%%%%%%%%%%%%%%
\begin{frame}[fragile]\frametitle{}
\begin{center}
{\Large Calculus}
\end{center}
\end{frame}

%%%%%%%%%%%%%%%%%%%%%%%%%%%%%%%%%%%%%%%%%%%%%%%%%%%%%%%%%%%
 \begin{frame}[fragile]\frametitle{}
\begin{itemize}
\item Algebra deals with finite processes.
\item Calculus deals with infinitesimal processes.
\item Limiting situations.
\item Computers can not handle, need to approximate.
\end{itemize}
\end{frame}



%%%%%%%%%%%%%%%%%%%%%%%%%%%%%%%%%%%%%%%%%%%%%%%%%%%%%%%%%%%
 \begin{frame}[fragile]\frametitle{Main parts of Calculus}
\begin{itemize}
\item Numbers: how the number-line (`x' axis for single variable) is constructed. Various sets with infinite items.
\item Functions: Functions that generate numbers, their types.
\item Most real-life sets are finite, e.g. number of people in the world, number of hair on head, etc.
\item But numbers are infinite.
\end{itemize}
\end{frame}



