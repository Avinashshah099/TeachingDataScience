%%%%%%%%%%%%%%%%%%%%%%%%%%%%%%%%%%%%%%%%%%%%%%%%%%%%%%%%%%%%%%%%%%%%%%%%%%%%%%%%%%
\begin{frame}[fragile]\frametitle{}
\begin{center}
{\Large Slots}

{\tiny (Ref: https://rasa.com/docs/rasa/core/slots/)}
\end{center}
\end{frame}

%%%%%%%%%%%%%%%%%%%%%%%%%%%%%%%%%%%%%%%%%%%%%%%%%%%%%%%%%%%
 \begin{frame}[fragile]\frametitle{What are slots?}
\begin{itemize}
\item Slots are your bot's memory.
\item User info is stored in them, a key-value store
\item External info is also stored in them (e.g. the result of a database query)
\item Most slot types influence how dialog progresses. Except one 'unfeaturized'
\end{itemize}
\end{frame}


%%%%%%%%%%%%%%%%%%%%%%%%%%%%%%%%%%%%%%%%%%%%%%%%%%%%%%%%%%%
 \begin{frame}[fragile]\frametitle{How Rasa Uses Slots?}
\begin{itemize}
\item The Policy doesn’t have access to the value of your slots. 
\item It receives a featurized representation.
\item The policy just sees a 1 or 0 depending on whether it is set.
\item You should choose your slot types carefully!
\item Depending on the type, featurization will happen.
\end{itemize}
\end{frame}


%%%%%%%%%%%%%%%%%%%%%%%%%%%%%%%%%%%%%%%%%%%%%%%%%%%%%%%%%%%
 \begin{frame}[fragile]\frametitle{Slots Set from NLU}
\begin{itemize}
\item If your NLU model picks up an entity, and your domain contains a slot with the same name, the slot will be set automatically. 
\item For example:
\begin{lstlisting}
# story_01
* greet{"name": "Ali"}
  - slot{"name": "Ali"}
  - utter_greet
\end{lstlisting}
\item In this case, you don’t have to include the \- slot{} part in the story, because it is automatically picked up.
\item To disable this behavior for a particular slot, you can set the auto\_fill attribute to False in the domain file:
\begin{lstlisting}
slots:
  name:
    type: text
    auto_fill: False
\end{lstlisting}
\end{itemize}
\end{frame}

%%%%%%%%%%%%%%%%%%%%%%%%%%%%%%%%%%%%%%%%%%%%%%%%%%%%%%%%%%%
 \begin{frame}[fragile]\frametitle{Slots Set By Clicking Buttons}
\begin{itemize}
\item You can specify in domain file like this: 
\begin{lstlisting}
utter_ask_color:
- text: "what color would you like?"
  buttons:
  - title: "blue"
    payload: '/choose{"color": "blue"}'
  - title: "red"
    payload: '/choose{"color": "red"}'
\end{lstlisting}
\end{itemize}
\end{frame}

%%%%%%%%%%%%%%%%%%%%%%%%%%%%%%%%%%%%%%%%%%%%%%%%%%%%%%%%%%%
 \begin{frame}[fragile]\frametitle{Slots Set by Actions}
\begin{itemize}
\item To set slots by returning events in custom actions
\item Stories need to include the slots. 
\item For example, you have a custom action to fetch a user’s profile, and you have a categorical slot called account\_type. 
\item When the fetch\_profile action is run, it returns a rasa.core.events.SlotSet event:
\begin{lstlisting}
slots:
   account_type:
      type: categorical
      values:
      - premium
      - basic
\end{lstlisting}
\end{itemize}
\end{frame}

%%%%%%%%%%%%%%%%%%%%%%%%%%%%%%%%%%%%%%%%%%%%%%%%%%%%%%%%%%%
 \begin{frame}[fragile]\frametitle{Slots Set by Actions}
 actions.py
\begin{lstlisting}
from rasa_sdk.actions import Action
from rasa_sdk.events import SlotSet
import requests

class FetchProfileAction(Action):
    def name(self):
        return "fetch_profile"

    def run(self, dispatcher, tracker, domain):
        url = "http://myprofileurl.com"
        data = requests.get(url).json
        return [SlotSet("account_type", data["account_type"])]
\end{lstlisting}
\end{frame}

%%%%%%%%%%%%%%%%%%%%%%%%%%%%%%%%%%%%%%%%%%%%%%%%%%%%%%%%%%%
 \begin{frame}[fragile]\frametitle{Slots Set by Actions}
stories.md
\begin{lstlisting}
# story_01
* greet
  - action_fetch_profile
  - slot{"account_type" : "premium"}
  - utter_welcome_premium

# story_02
* greet
  - action_fetch_profile
  - slot{"account_type" : "basic"}
  - utter_welcome_basic
\end{lstlisting}
\end{frame}