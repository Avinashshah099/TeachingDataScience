%%%%%%%%%%%%%%%%%%%%%%%%%%%%%%%%%%%%%%%%%%%%%%%%%%%%%%%%%%%%%%%%%%%%%%%%%%%%%%%%%%
\begin{frame}[fragile]\frametitle{}
\begin{center}
{\Large Exceptions}
\end{center}
\end{frame}


%%%%%%%%%%%%%%%%%%%%%%%%%%%%%%%%%%%%%%%%%%%%%%%%%%%%%%%%%%%%%%%%%%%%%%%%%%%%%%%%%%%
\begin{frame}[fragile]\frametitle{Exceptions}

\begin{itemize}
\item ``Exceptions'' are error conditions.
\item Code that intercepts some error conditions and
  reacts.
\item  Inherits from the built-in  Exception class.

\begin{lstlisting}
 class NewKindOfError(Exception):
    pass
 \end{lstlisting}

\item
   Exceptions are handled by class name, so they usually do not need
   any new methods (although you are free to define some if needed).

\end{itemize}
\end{frame}

%%%%%%%%%%%%%%%%%%%%%%%%%%%%%%%%%%%%%%%%%%%%%%%%%%%%%%%%%%%%%%%%%%%%%%%%%%%%%%%%%%%
\begin{frame}[fragile] \frametitle{What does an exception look like?}
\begin{lstlisting}
>>> stream.write('foo')
Traceback (most recent call last):
  File "<stdin>", line 1, in <module>
IOError: File not open for writing
\end{lstlisting}

  This is the exception \emph{message}: it is supposed to be read
  by the (human) user.
\end{frame}

%%%%%%%%%%%%%%%%%%%%%%%%%%%%%%%%%%%%%%%%%%%%%%%%%%%%%%%%%%%%%%%%%%%%%%%%%%%%%%%%%%%
\begin{frame}[fragile]\frametitle{Syntax}
\begin{lstlisting}
try:
  # code that might raise an exception
except SomeException:
  # handle some exception
except AnotherException as ex:
  # the actual Exception instance
  # is available as variable `ex'
finally:
  # performed on exit in any case
\end{lstlisting}

  The optional finally clause is executed on exit from the
  try or except block in \emph{any} case.
\end{frame}


%%%%%%%%%%%%%%%%%%%%%%%%%%%%%%%%%%%%%%%%%%%%%%%%%%%%%%%%%%%%%%%%%%%%%%%%%%%%%%%%%%%
\begin{frame}[fragile]\frametitle{Raising exceptions}

\begin{itemize}
\item   Use the raise statement with an \lstinline{Exception}
  instance:
\begin{lstlisting}
if an_error_occurred:
  raise AnError("Spider sense is tingling.")
 
\end{lstlisting}

\item The exception class name specifies what kind of exception to raise.

\item   You must use an \emph{existing} exception class; we shall learn
  how to define new exceptions in the object-orientation lectures.

\item   The exception message is an arbitrary string.  It is meant for
  humans to read, so try to describe the error condition clearly and
  concisely.
\end{itemize}

 
\end{frame}

%%%%%%%%%%%%%%%%%%%%%%%%%%%%%%%%%%%%%%%%%%%%%%%%%%%%%%%%%%%%%%%%%%%%%%%%%%%%%%%%%%%
\begin{frame}[fragile]\frametitle{Raising exceptions}
  Within an except clause, you can use raise
  with no arguments to re-raise the current exception:
\begin{lstlisting}
try:
  something()
except ItDidntWork:
  do_cleanup()
  # re-raise exception to caller
  raise
\end{lstlisting}
\end{frame}

%%%%%%%%%%%%%%%%%%%%%%%%%%%%%%%%%%%%%%%%%%%%%%%%%%%%%%%%%%%%%%%%%%%%%%%%%%%%%%%%%%%
\begin{frame}[fragile]\frametitle{raise statement}
Throw or raise an exception.
\begin{itemize}
\item Forms:
	\begin{itemize}
	\item \lstinline{raise instance}
	\item \lstinline{raise MyExceptionClass(value)}  -- preferred.
	\item \lstinline{raise MyExceptionClass, value}
	\end{itemize}
\item The  \lstinline{raise}  statement takes:
	\begin{itemize}
	\item An (instance of) a built-in exception class.
	\item An instance of class   \lstinline{Exception}  or
	\item An instance of a built-in subclass of class   \lstinline{Exception}  or
	\item An instance of a user-defined subclass of class   \lstinline{Exception}  or
	\item One of the above classes and (optionally) a value (for example, a string or a tuple).
	\end{itemize}
\end{itemize}
\end{frame}



%%%%%%%%%%%%%%%%%%%%%%%%%%%%%%%%%%%%%%%%%%%%%%%%%%%%%%%%%%%%%%%%%%%%%%%%%%%%%%%%%%%
\begin{frame}[fragile]\frametitle{Exercise}
Write a function to compute 5/0 and use try/except to catch the exceptions.
\end{frame}

%%%%%%%%%%%%%%%%%%%%%%%%%%%%%%%%%%%%%%%%%%%%%%%%%%%%%%%%%%%%%%%%%%%%%%%%%%%%%%%%%%%
\begin{frame}[fragile]\frametitle{Solution}
\begin{lstlisting}
def throws():
    return 5/0

try:
    throws()
except ZeroDivisionError:
    print "division by zero!"
except Exception as err:
    print 'Caught an exception'
finally:
    print 'In finally block for cleanup'
\end{lstlisting}
\end{frame}



%%%%%%%%%%%%%%%%%%%%%%%%%%%%%%%%%%%%%%%%%%%%%%%%%%%%%%%%%%%%%%%%%%%%%%%%%%%%%%%%%%%
\begin{frame}[fragile]\frametitle{Exercise}
Define a custom exception class which takes a string message as attribute.

Hints:

To define a custom exception, we need to define a class inherited from Exception.
\end{frame}

%%%%%%%%%%%%%%%%%%%%%%%%%%%%%%%%%%%%%%%%%%%%%%%%%%%%%%%%%%%%%%%%%%%%%%%%%%%%%%%%%%%
\begin{frame}[fragile]\frametitle{Solution}
\begin{lstlisting}
class MyError(Exception):
    """My own exception class

    Attributes:
        msg  -- explanation of the error
    """

    def __init__(self, msg):
        self.msg = msg

error = MyError("something wrong")
\end{lstlisting}
\end{frame}


% %%%%%%%%%%%%%%%%%%%%%%%%%%%%%%%%%%%%%%%%%%%%%%%%%%%%%%%%%%%%%%%%%%%%%%%%%%%%%%%%%%%
% \begin{frame}[fragile]\frametitle{Exercise}
% \begin{itemize}
% \item Enhance the function parse\_data defined in the ``Warm-up''
% \item Exercise: if an error is found in the data read from file,
    % silently ignore it and drop the line.

% \item Test your code on the
    % \href{https://raw.github.com/gc3-uzh-ch/python-course/master/rt.tsv}{\lstinline{rt.dirty.csv}}
    % data file: your code should detect and ignore the 4 malformed
    % lines in the file.
% \end{itemize}
% \end{frame}
